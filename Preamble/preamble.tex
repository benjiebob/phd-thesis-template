% ******************************************************************************
% ****************************** Custom Margin *********************************

% Add `custommargin' in the document class options to use this section
% Set {innerside margin / outerside margin / topmargin / bottom margin}  and
% other page dimensions
\ifsetCustomMargin
  \RequirePackage[left=37mm,right=30mm,top=35mm,bottom=30mm]{geometry}
  \setFancyHdr % To apply fancy header after geometry package is loaded
\fi

% Add spaces between paragraphs
%\setlength{\parskip}{0.5em}
% Ragged bottom avoids extra whitespaces between paragraphs
\raggedbottom
% To remove the excess top spacing for enumeration, list and description
%\usepackage{enumitem}
%\setlist[enumerate,itemize,description]{topsep=0em}

% *****************************************************************************
% ******************* Fonts (like different typewriter fonts etc.)*************

% Add `customfont' in the document class option to use this section

\ifsetCustomFont
  % Set your custom font here and use `customfont' in options. Leave empty to
  % load computer modern font (default LaTeX font).
  %\RequirePackage{helvet}

  % For use with XeLaTeX
  %  \setmainfont[
  %    Path              = ./libertine/opentype/,
  %    Extension         = .otf,
  %    UprightFont = LinLibertine_R,
  %    BoldFont = LinLibertine_RZ, % Linux Libertine O Regular Semibold
  %    ItalicFont = LinLibertine_RI,
  %    BoldItalicFont = LinLibertine_RZI, % Linux Libertine O Regular Semibold Italic
  %  ]
  %  {libertine}
  %  % load font from system font
  %  \newfontfamily\libertinesystemfont{Linux Libertine O}
\fi

% *****************************************************************************
% **************************** Custom Packages ********************************

% ************************* Algorithms and Pseudocode **************************

%\usepackage{algpseudocode}


% ********************Captions and Hyperreferencing / URL **********************

% Captions: This makes captions of figures use a boldfaced small font.
%\RequirePackage[small,bf]{caption}

\RequirePackage[labelsep=space,tableposition=top]{caption}
\renewcommand{\figurename}{Fig.} %to support older versions of captions.sty


% *************************** Graphics and figures *****************************

%\usepackage{rotating}
%\usepackage{wrapfig}

% Uncomment the following two lines to force Latex to place the figure.
% Use [H] when including graphics. Note 'H' instead of 'h'
%\usepackage{float}
%\restylefloat{figure}

% Subcaption package is also available in the sty folder you can use that by
% uncommenting the following line
% This is for people stuck with older versions of texlive
%\usepackage{sty/caption/subcaption}
\usepackage[table,xcdraw]{xcolor}
\usepackage{subcaption}

% ********************************** Tables ************************************
\usepackage{booktabs} % For professional looking tables
\usepackage{multirow}

%\usepackage{multicol}
%\usepackage{longtable}
%\usepackage{tabularx}


% *********************************** SI Units *********************************
\usepackage{siunitx} % use this package module for SI units


% ******************************* Line Spacing *********************************

% Choose linespacing as appropriate. Default is one-half line spacing as per the
% University guidelines

% \doublespacing
% \onehalfspacing
% \singlespacing


% ************************ Formatting / Footnote *******************************

% Don't break enumeration (etc.) across pages in an ugly manner (default 10000)
%\clubpenalty=500
%\widowpenalty=500

%\usepackage[perpage]{footmisc} %Range of footnote options


% *****************************************************************************
% *************************** Bibliography  and References ********************

%\usepackage{cleveref} %Referencing without need to explicitly state fig /table

% Add `custombib' in the document class option to use this section
\ifuseCustomBib
   \RequirePackage[square, sort, numbers, authoryear]{natbib} % CustomBib

% If you would like to use biblatex for your reference management, as opposed to the default `natbibpackage` pass the option `custombib` in the document class. Comment out the previous line to make sure you don't load the natbib package. Uncomment the following lines and specify the location of references.bib file

%\RequirePackage[backend=biber, style=numeric-comp, citestyle=numeric, sorting=nty, natbib=true]{biblatex}
%\addbibresource{References/references} %Location of references.bib only for biblatex, Do not omit the .bib extension from the filename.

\fi

% changes the default name `Bibliography` -> `References'
\renewcommand{\bibname}{References}


% ******************************************************************************
% ************************* User Defined Commands ******************************
% ******************************************************************************

% *********** To change the name of Table of Contents / LOF and LOT ************

%\renewcommand{\contentsname}{My Table of Contents}
%\renewcommand{\listfigurename}{My List of Figures}
%\renewcommand{\listtablename}{My List of Tables}


% ********************** TOC depth and numbering depth *************************

\setcounter{secnumdepth}{2}
\setcounter{tocdepth}{2}


% ******************************* Nomenclature *********************************

% To change the name of the Nomenclature section, uncomment the following line

%\renewcommand{\nomname}{Symbols}


% ********************************* Appendix ***********************************

% The default value of both \appendixtocname and \appendixpagename is `Appendices'. These names can all be changed via:

%\renewcommand{\appendixtocname}{List of appendices}
%\renewcommand{\appendixname}{Appndx}

% *********************** Configure Draft Mode **********************************

% Uncomment to disable figures in `draft'
%\setkeys{Gin}{draft=true}  % set draft to false to enable figures in `draft'

% These options are active only during the draft mode
% Default text is "Draft"
%\SetDraftText{DRAFT}

% Default Watermark location is top. Location (top/bottom)
%\SetDraftWMPosition{bottom}

% Draft Version - default is v1.0
%\SetDraftVersion{v1.1}

% Draft Text grayscale value (should be between 0-black and 1-white)
% Default value is 0.75
%\SetDraftGrayScale{0.8}


% ******************************** Todo Notes **********************************
%% Uncomment the following lines to have todonotes.

%\ifsetDraft
%	\usepackage[colorinlistoftodos]{todonotes}
%	\newcommand{\mynote}[1]{\todo[author=kks32,size=\small,inline,color=green!40]{#1}}
%\else
%	\newcommand{\mynote}[1]{}
%	\newcommand{\listoftodos}{}
%\fi

% Example todo: \mynote{Hey! I have a note}

% ******************************** Highlighting Changes **********************************
%% Uncomment the following lines to be able to highlight text/modifications.
%\ifsetDraft
%  \usepackage{color, soul}
%  \newcommand{\hlc}[2][yellow]{{\sethlcolor{#1} \hl{#2}}}
%  \newcommand{\hlfix}[2]{\texthl{#1}\todo{#2}}
%\else
%  \newcommand{\hlc}[2]{}
%  \newcommand{\hlfix}[2]{}
%\fi

% Example highlight 1: \hlc{Text to be highlighted}
% Example highlight 2: \hlc[green]{Text to be highlighted in green colour}
% Example highlight 3: \hlfix{Original Text}{Fixed Text}

% *****************************************************************************
% ******************* Better enumeration my MB*************
\usepackage{enumitem}

% BJB Extra Packages

\usepackage{tikz}
% \usepackage{subfig}

% \usepackage{xcolor}
\usepackage{floatrow}


\usepackage[export]{adjustbox} % for fixing max width of images
\usepackage{tabularx}
\usepackage{hyperref}
\usepackage{cleveref}
\usepackage{wrapfig}

\usepackage{xspace}
\usepackage{url}
\usepackage{nicefrac}

\usepackage{graphicx}
\usepackage{comment}
\usepackage{amsmath,amssymb} % define this before the line numbering.
\usepackage{color}
\usepackage{booktabs}
\usepackage{cuted}
\usepackage{array}
\usepackage{ragged2e}
\usepackage{amsthm}
\usepackage{etoolbox}
\usepackage{optidef}
% \usepackage{kbordermatrix}

\usepackage{longtable}

\newcommand{\citetodo}[2]{[TODO: #1]}
% \newcommand{\R}{\mathbb{R}}

\def\R#1{{\mathbb{R}^{#1}}}
\def\RR#1#2{{\mathbb{R}^{#1 \times #2}}}
\def\posn{\phi}
\def\pose{\theta}
\def\npose{P}
\def\shape{\beta}
\def\shapescale{\delta}
\def\nshape{B}
\def\verts{\nu}
\def\nverts{V}
\def\jointselect{\mathtt{K}}
\def\njoints{J}
\def\proj{\pi}
\def\bvec#1{\bar{#1}}

\def\lterm#1{\subsubsection{$\LL{#1}$:}}

% \def\E#1{{E_{\text{#1}}}}
\def\E#1{E_\textrm{#1}}

\def\ss#1{\vspace{-0ex}\subsubsection{#1}}

\def\L#1{L_{\textrm{#1}}}
% \def\LL#1{L_{\text{#1}}}
\def\LL#1{\L{#1}}

\def\R#1{{\mathbb{R}^{#1}}}
\def\RR#1#2{{\mathbb{R}^{#1 \times #2}}}
\def\RRR#1#2#3{{\mathbb{R}^{#1 \times #2 \times #3}}}

\def\scale{\kappa}
\def\trans{t}
% \def\betacov{{\Sigma_{\beta}}}
% \def\posecov{{\Sigma_{\pose}}}
% \def\posemean{{\mu_{\pose}}}
% \def\betamean{{\mu_{\beta}}}
\def\nimages{N}
\def\nshape{B}

\def\shapecov{\Sigma_{\shape}}
\def\shapemu{\mu_{\shape}}
\def\shapepi{\Pi_{\shape}}
\def\scalecov{\Sigma_{\scale}}
\def\scalemu{\mu_{\scale}}
\def\scalepi{\Pi_{\scale}}
\def\shapescalecov{\Sigma_{\shapescale}}
\def\shapescalemu{\mu_{\shapescale}}
\def\shapescalepi{\Pi_{\shapescale}}

\def\inputjtstwo{X^{*}}
\def\inputjtsthree{Y^{*}}
\def\inputsil{S}

\def\W#1{\lambda_\textrm{#1}}

\def\f{f}

\def\lazycite#1#2{[#2]}

\DeclareMathOperator\SMPL{SMPL}
\DeclareMathOperator\SMAL{SMAL}

\newcommand\inv[1]{\left.#1\right.^{-1}}
\newcommand\transpose[1]{\left.#1\right.^{T}}
\newcommand{\anote}[1]{{\color{red}[#1]}}


\theoremstyle{definition}
\newtheorem{definition}{Definition}[section]

\newtheorem{theorem}{Theorem}[section]

\newfloatcommand{capbtabbox}{table}[][\FBwidth]

\def\bb{\rule{2in}{0pt}\rule{0pt}{1in}}
\def\labelledpic#1#2{
\begin{tikzpicture}
\node (pic) {#1};
\path[fill=white,draw=gray,thick] (pic.south west) +(3ex,3ex) circle (2ex)
node {#2};
\end{tikzpicture}
}

\def\seq#1#2#3#4{\left[{#1_{#2}}\right]_{#2=#3}^{#4}}
\newcommand\Set[2]{\{\,#1\mid#2\,\}}
\newcommand\SET[2]{\Set{#1}{\text{#2}}}

\DeclareMathAlphabet{\mathcal}{OMS}{cmsy}{m}{n}

\DeclareMathOperator*{\argmax}{arg\,max}
\DeclareMathOperator*{\argmin}{arg\,min}

\usepackage{sty/tkz-graph}
\GraphInit[vstyle = Shade]
\tikzset{
  LabelStyle/.style = { rectangle, rounded corners, draw,
                        minimum width = 2em, fill = yellow!50,
                        text = red, font = \bfseries },
  VertexStyle/.append style = { inner sep=5pt,
                                font = \Large\bfseries},
  EdgeStyle/.append style = {->, bend left} }