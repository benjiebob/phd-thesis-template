%!TEX root = ../thesis.tex
%*******************************************************************************
%****************************** Third Chapter **********************************
%*******************************************************************************
\chapter{End-to-end Dog Shape Recovery with a Learned Shape Prior}\label{chap:wldo}

% **************************** Define Graphics Path **************************
\ifpdf
    \graphicspath{{Chapter5/Figs/Raster/}{Chapter5/Figs/PDF/}{Chapter5/Figs/}}
\else
    \graphicspath{{Chapter5/Figs/Vector/}{Chapter5/Figs/}}
\fi


% Plan:
% - Introduction
%   - discuss the dog category, its been little explored, link to human reconstruction (infants, overweight individuals)
%   - We also want to make it run real-time.
% - Related Work - focus on techniques which focus on challenging caetgories. Draw out the most similar work (kanazawa birds and zebras). For this work, we want to only deal with monocular input to extend applicability to multiple categories. "in-the-loop" methods.
% - Technical related work [AWF] - Methods for adapting priors, expectation maximization etc.
% - StanfordExtra - since in this paper we want high fidelity reconstructions for one category, it is economical to collect training data for it. 
% - SMBLD - Design of a realistic dog template model. Possibly run SMAL optimizer using this SMBLD model and compare with previous paper. While better, there is still room for improvement.
% - Expectation maximization in the loop.
% - End-to-end deep learning approach. Architecture etc.
% - Experimentation. 


% Possible Additional Experiments
% (1) Add a model fitting stage
% (2) Add texture
% (3) In theory could also run on horses, but would take time.

\section{Introduction}

This chapter introduces an automatic, end-to-end method for recovering the 3D pose and shape of dogs from monocular internet images. The large variation in shape between dog breeds, significant occlusion and low quality of internet images makes this a challenging problem. In addition, dogs are poorly represented in existing 3D morphable models. Despite the prevelance of dogs in society (there are more than 63 million pet dogs in the US alone~\cite{appa20}), these factors have perhaps contributed to the lack of effective 3D reconstruction methods for dogs. This work learns a richer prior over dog shapes than in previous work which helps regularize parameters estimation. This chapter demonstrates that the natural variation present in a 2D dataset is sufficient to learn a detailed 3D prior by incorporating expectation maximization (EM) steps into the training process. It is also demonstrated how to adapt the raw linear blend skinning process to include additional limb scaling parameters, which are shown to improve the quality of reconstruction. Inspired by recent methods for human reconstruction ~\cite{kanazawa18end-to-end}, the system comprises an end-to-end convolutional neural network which directly predicts the morphable model parameters with no subsequent energy minimization phase, enabling real-time operation. This is important for potential downstream tasks, such as behaviour or health monitoring systems which should respond immediately to identified causes of concern. At present, state-of-the-art animal reconstruction techniques incoporate per-image (or per-sequence as in \Cref{chap:cgas}) test-time energy minimization procedures that prohibit real-time application. In comparison, the shape prior refinement process introduced here is applied only at training time and leads to improved shape reconstructions enabling high quality 3D dog predictions that make a subsequent refinement step optional. The experimentation section is based on StanfordExtra, a new `in the wild' dataset of dog images, containing 120 breeds. The method presented achieves state-of-the-art performance, and outperforms model fitting approaches, even when they are given access to ground truth annotations at test time. 

% WHY DOGS???
%A particular species of interest is the dog, however it is noticeable that existing work has not yet demonstrated effective 3D reconstruction of dogs over large test sets. We postulate that this is partially because dog breeds are remarkably dissimilar in shape and texture, presenting a challenge to the current state of the art.

% 

% The hope is to overcome limitations surrounding the practical application of such systems. For example, poor quality single-view shape reconstructions can lead to missed health and wellbeing insights, can lead to animal researchers are unable to analyse subtle differences in weight and body proportions which could otherwise lead to health and wellbeing insights. 

%A by-product of training, we generate a new parameterized model (including limb scaling) SMBLD which we release alongside our new annotation dataset StanfordExtra to the research community.

% Approach is weakly supervised deep network (i.e. we have no 3D training data)
% We adapt our 3D prior to be multi-modal
% We learn a 3D prior using a 2D image dataset
% Inspired by SPIN: this prior is learned using an optimization process during training time
% We adapt our 3D model using scale parameters
% Why dogs??

% WORK IN some lit review stuff from the paper.

% \subsection{Animal datasets}

% To learn a representative 3D prior and enable quantitative comparison to existing state-of-the-art methods, a large 2D animal dataset is required. Whereas existing methods typically evaluate on a few examples from multiple species, this work focuses solely on the dog category. Although focusing on a single category limits the overall shape diversity, the shape variation between breeds is significant and is more complex. Capturing subteleties between multiple Capturing these subtleties 

\subsection{Learning a 3D animal prior}

As discussed in \Cref{chap:relwork}, there is a long history of 3D reconstruction approaches that rely on morphable models. Earlier approaches~\lazycite{examples}{examples} follow an energy minimization framework, in which a 3D template is optimized to match a single image (or sequence). However, more recent works tackle the reconstruction problem as a direct regression from the input image to 3D model parameters. For humans, deep learning approaches typically 3D morphable models (e.g. SMPL), 3D scans for learning priors, \emph{paired datasets} containing images and corresponding 3D annotations, and \emph{unpaired datasets} containing images and 2D annotations. Methods of this kind have now begun to outperform optimization-based approaches (e.g. SMPLify~\lazycite{SMPLify}{SMPLify}). A notable property of these ``end-to-end'' approaches, (i.e. deep learning only with no test-time optimization phase) is that they are able to operate in real-time. Occasionally, these methods are also evaluated in an `unpaired' setting, in which the paired datasets are ommited (or only 2D annotations are used). In such an unpaired setting, techniques must overcome fundamental ambiguities in the 2D-to-3D mapping, which can otherwise result in predicted bodies with impossible joint angles or are extremely skinny. This is typically achieved using learnt 3D priors, which ensure human body predictions lie on the manifold of plausible bodies. These two properties have particular relevance to the 3D dog reconstruction task of this chapter. Firstly, real-time performance is critical to enable future downstream tasks such as behaviour analysis, in which causes for concern must be identified immediately. Secondly, the lack of available paired animal datasets mandates training in an unpaired setting. 

Existing approaches in animal reconstruction are mostly optimization based. These methods typically design an energy function that balances \emph{data terms} to encourage strong alignment between the 3D model and input image, and \emph{prior terms} which ensure realistic predictions. Due to the limited 3D training data available for animals, existing priors are built using a few toy figurines rather than scans of real subjects such as used for SMPL~\lazycite{SMPL}{SMPL}. Consequently, existing priors are of much lower fidelity and poorly represent some species, particularly dogs. This leads to poor performance by state-of-the-art single-view methods on the dog category, since the prior term tends to be too restrictive. Included among these is the method introduced in \Cref{chap:cgas} in which synthetic data is generated according to the fixed 3D prior, preventing strong generalization to this category. % REWORD

Some recent deep learning methods overcome this by forgoing the data-driven 3D shape prior altogether. SMALST is one such example; they instead train a deep network on 3D synthetic data generated using video sequences. CMU and UCMU reconstruct birds without a 3D morphable model, and replace a data-driven 3D shape prior with smoothness terms, deformation constraints and symmetry constraints. However, birds are treated in this work as an unarticulated category, which leads to simpler optimization. 

Of course, the quality of 3D animal priors (and reconstructions) could be improved by collecting larger and higher quality 3D scans with which to build 3D morphable models and associated shape priors. However, collecting such a dataset typically requires an expensive and time-consuming scanning process for real animal subjects, or labour intensive process by talented 3D graphics artists to generate multiple animal models. This chapter argues that even a low-quality unimodal 3D shape prior can act as a useful initialization for a novel \emph{refinement process} which learns a more expressive multimodal prior from an annotated 2D dataset.

\subsection{Refinement steps ``in the loop''}

A key insight relied upon in this chapter is the potential for collaboration between a deep neural network which processes input images to regress 3D model parameters and an optimization process that tunes the 3D prior. The technique used for this was inspired by the SPIN network of Kolotouros et al. who introduced a 3D human reconstruction approach based on a similar hybrid. In their approach, the HMR backbone~\lazycite{HMR}{HMR} is used to yield a set of SMPL~\lazycite{SMPL}{SMPL} human body shape and pose parameters $\Theta_{reg}$ from an input image. During SPIN's training phase, the initial fit $\Theta_{reg}$ is processed by the SMPLify~\lazycite{SMPLify}{SMPLify} model fitting procedure, which refines the initial fit based on the ground truth joints to yield a new estimate $\Theta_{opt}$. The difference between these two predictions is then expressed as a loss $||\Theta_{reg} - \Theta_{opt}||$ which is backpropagated through the regression network to help it generate improved model parameter predictions. A notable property of this architecture is that, since the energy minimization phase is used only at training time, the test time performance is unaffected.

A similar strategy is relied upon in this work, to enable procedural adaptations to an initial 3D prior to enable accurate and real-time 3D dog reconstructions. Inspired by SPIN, the network's training phase incoporates a optimization strategy based on expectation maximization which regularly updates the 3D shape prior based on weights learned from an annotated 2D image dataset. This creates a collaborative effect: the representational power of the 3D shape prior gradually improves as the 3D reconstruction network produces better predictions and vice versa. Experimentally, it is shown that the reconstructed dog models are of good enough quality to avoid a time consuming test time optimization process. Further details on the expectation maximization process and overall reconstruction network are deferred to the later method section.


\begin{figure}[t]
\setlength{\fboxsep}{0pt}%
\setlength{\fboxrule}{0pt}%

% Define left and right aligned fixed width columns
\renewcommand\tabularxcolumn[1]{m{#1}}% for vertical centering text in X column
\newcolumntype{L}[1]{>{\hsize=#1\hsize\raggedright\arraybackslash}X}%
\renewcommand\tabularxcolumn[1]{m{#1}}% for vertical centering text in X column
\newcolumntype{R}[1]{>{\hsize=#1\hsize\raggedleft\arraybackslash}X}%

\begin{tabularx}{1\textwidth}{@{} *{3}{R{0.1666}L{0.1666}}@{}p{0cm} @{}}

    \includegraphics[height=0.2\linewidth, max width=0.15\linewidth]{ours_sup/n02088632-bluetick/orig/n02088632_744_crop.jpg} &
    \includegraphics[height=0.2\linewidth, max width=0.15\linewidth]{ours_sup/n02088632-bluetick/fit/n02088632_744_crop.jpg} &

    \includegraphics[height=0.2\linewidth, max width=0.15\linewidth, max width=0.15\linewidth]{ours_sup/n02087394-Rhodesian_ridgeback/orig/n02087394_5552_crop.jpg} &
    \includegraphics[height=0.2\linewidth, max width=0.15\linewidth, max width=0.15\linewidth]{ours_sup/n02087394-Rhodesian_ridgeback/fit/n02087394_5552_crop.jpg} &

    \includegraphics[height=0.2\linewidth, max width=0.15\linewidth]{ours_sup/n02108422-bull_mastiff/orig/n02108422_4039_crop.jpg} &
    \includegraphics[height=0.2\linewidth, max width=0.15\linewidth]{ours_sup/n02108422-bull_mastiff/fit/n02108422_4039_crop.jpg} &



    \\

    \includegraphics[height=0.2\linewidth, max width=0.15\linewidth]{ours_sup/n02099429-curly-coated_retriever/orig/n02099429_2570_crop.jpg} &
    \includegraphics[height=0.2\linewidth, max width=0.15\linewidth]{ours_sup/n02099429-curly-coated_retriever/fit/n02099429_2570_crop.jpg} &

    \includegraphics[height=0.2\linewidth, max width=0.15\linewidth]{ours_sup/n02091244-Ibizan_hound/orig/n02091244_3373_crop.jpg} &
    \includegraphics[height=0.2\linewidth, max width=0.15\linewidth]{ours_sup/n02091244-Ibizan_hound/fit/n02091244_3373_crop.jpg} &

    \includegraphics[height=0.2\linewidth, max width=0.15\linewidth]{ours_sup/n02087046-toy_terrier/orig/n02087046_133_crop.jpg} &
    \includegraphics[height=0.2\linewidth, max width=0.15\linewidth]{ours_sup/n02087046-toy_terrier/fit/n02087046_133_crop.jpg} &

    \\
    \includegraphics[height=0.2\linewidth, max width=0.15\linewidth]{ours_sup/n02097658-silky_terrier/orig/n02097658_6672.jpg} &
    \includegraphics[height=0.2\linewidth, max width=0.15\linewidth]{ours_sup/n02097658-silky_terrier/fit/n02097658_6672.jpg} &

    \includegraphics[height=0.2\linewidth, max width=0.15\linewidth]{ours_sup/n02110806-basenji/orig/n02110806_2957_crop.jpg} &
    \includegraphics[height=0.2\linewidth, max width=0.15\linewidth]{ours_sup/n02110806-basenji/fit/n02110806_2957_crop.jpg} &

    \includegraphics[height=0.2\linewidth, max width=0.15\linewidth]{ours_sup/n02089973-English_foxhound/orig/n02089973_1763_crop.jpg} &
    \includegraphics[height=0.2\linewidth, max width=0.15\linewidth]{ours_sup/n02089973-English_foxhound/fit/n02089973_1763_crop.jpg}  &  

\end{tabularx}\medbreak
\caption{
\textbf{End-to-end Dog Shape Recovery with a Learned Shape Prior.}
We propose a novel method that, given a monocular image of a dog can predict a set of parameters for our SMBLD 3D dog model which is consistent with the input. We regularize learning using a multi-modal shape prior, which is tuned during training with an expectation maximization scheme.\label{fig:splash}}
\end{figure}

\subsection{Contributions}

The method proposed extends the state of the art in several ways.
While each of these qualities exist in some existing works, we believe ours is the first to exhibit this combination, leading to a new state of the art in terms of scale and object diversity.

\begin{enumerate}
    \item We reconstruct pose and shape on a test set of 1703 low-quality internet images of a complex 3D object class (dogs).
    \item We directly regress to object pose and shape from a single image without a model fitting stage.
    \item We use easily obtained 2D annotations in training, and none at test time.
    \item We incorporate fitting of a new multi-modal prior into the training phase (via EM update steps), rather than fitting it to 3D data as in previous work.
    \item We introduce new degrees of freedom to the SMAL model, 
    allowing explicit scaling of subparts.
\end{enumerate}

\begin{figure*}[h]
    \centering
    \includegraphics[width=\textwidth]{OllieFigs/system_overview_cr.pdf}
    \caption{Our method consists of (1) a deep CNN encoder which condenses the input image into a feature vector (2) a set of prediction heads which generate SMBLD parameters for shape $\beta$, pose $\theta$, camera focal length $f$ and translation $t$ (3) skinning functions $F_v$ and $F_J$ which construct the mesh from a set of parameters, and (4) loss functions which minimise the error between projected and ground truth joints and silhouettes. Finally, we incorporate a mixture shape prior (5) which regularises the predicted 3D shape and is iteratively updated during training using expectation maximisation. At test time, our system (1) condenses the input image, (2) generates the SMBLD parameters and (3) constructs the mesh.}
    \label{fig:sys_overview_train_sup}
\end{figure*}
    

\section{Preliminaries}

\subsection{PCA shape space}

For shape spaces shape space formulations based on principal component analysis (PCA), recall the linear generator function $g: \R{d} \mapsto \R{3n}$ which maps a $d$-dimensional parameter space to $n$ 3D morphable model vertex coordinates: 

\begin{equation}
    g(w) = \bar{c} + Ew
\end{equation}

In this formulation, $\bar{c} \in \R{3n}$ is the mean 3D shape from a training dataset, and $E \in \R{3n \times d}$ is a matrix containing the $d$ most dominant eigenvectors computed over shape residuals $\{c_i - \bar{c_i}\}$. 

%% PCA does assumlaxe normal distribution of features See p.55 SAS book1 or Rummel, 19702 or Mardia, 19793.

A consequence of PCA construction is that the features in the $d$-dimensional parameter space follow a multivariate normal distribution. %% ASK ANDREW, or be less lazy and do the math
With this construction, one can define a likelihood function which measures the probability of a given shape vector $w \in \R{d}$

\begin{equation}
    f(w) = (2\pi)^{-\frac{d}{2}}\det(\Sigma)^{-\frac{1}{2}}e^{-\frac{1}{2}(w-\bar{c})^T\Sigma^{-1}(w-\bar{c})}
\end{equation}

For problems which aim to optimize $w$, the 3D shape prior is obtained by maximizing $f(w)$ or equivalently, by minimizing the negative log likelihood

\begin{equation}
     -\ln(f(w)) = -\frac{1}{2}\left[\ln\det(\Sigma) + d\ln(2\pi) +  (w - \bar{c})^T\Sigma^{-1}(w-\bar{c})\right]
\end{equation}
and dropping terms with no dependency on $w$ (which remain constant during optimization) leaves the Mahalanobis distance of $w$ to the origin

\begin{equation}
    L(w) = (w - \bar{c})^T\Sigma^{-1}(w-\bar{c})
\end{equation}


the 3D shape optimization problems, the 3D shape prior 

This construction assumes 3D faces in this $d$-dimensional parameter space follow a multivariate normal distribution (a design decision explored further in this thesis \Cref{chap:wldo}). In addition, the function $f(w)$ which defines the likelihood shape space vector $\alpha$ represents a plausible face, is therefore given by the Mahalanbois distance of $\alpha$ to the origin. 


\section{Building SMBLD: a new parametric dog model}

% Explain why the SMAL parameteric model is unsuitable for the dog category.

At the heart of our method is a parametric representation of a 3D animal mesh, which is based on the Skinned Multi-Animal Linear (SMAL) model proposed by~\cite{zuffi2017menagerie}. SMAL is a deformable 3D animal mesh parameterized by shape and pose. The \emph{shape}~$\shape \in \R\nshape$ parameters are PCA coefficients of an undeformed template mesh with limbs in default position. The \emph{pose}~$\pose \in \R\npose$ parameters meanwhile govern the joint angle rotations ($35 \times 3$ Rodrigues parameters) which effect the articulated limb movement. The model consists of a linear blend skinning function $F_{v}: (\pose, \shape) \mapsto V$, which generates a set of vertex positions $V \in \RR{3889}{3}$, and a joint function $F_{J}: (\pose, \shape) \mapsto J$, which generates a set of joint positions $J \in \RR{35}{3}$.

\subsection{Introducing scale parameters}
While SMAL has been shown to be adequate for representing a variety of quadruped types, we find that the modes of dog variation are poorly captured by the current model. This is unsurprising, since SMAL used only four dogs in its construction.

We therefore introduce a simple but effective way to improve the model's representational power over this particularly diverse  animal category. We augment the set of shape parameters $\beta$ with an additional set $\scale$ which independently scale parts of the mesh. For each model joint, we define parameters ${\scale_x,\scale_y,\scale_z}$ which apply a local scaling of the mesh along the local coordinate $x, y, z$ axes, before pose is applied. Allowing each joint to scale entirely independently can however lead to unrealistic deformations, so we share scale parameters between multiple joints, e.g. leg lengths. The new Skinned Multi-Breed Linear Model for Dogs (SMBLD) is therefore adapted from SMAL by adding $6$ scale parameters to the existing set of shape parameters. Figure~\ref{fig:shape_variation} shows how introducing scale parameters increases the flexibility of the SMAL model. We also extend the provided SMAL shape prior (which later initializes our EM procedure) to cover the new scale parameters by fitting SMBLD to a set of $13$ artist-designed 3D dog meshes. Further details left to the supplementary.

\begin{figure*}[t!]
    \centering
    % \includegraphics[width=0.23\linewidth]{OllieFigs/mean.png}
    % \includegraphics[width=0.23\linewidth]{OllieFigs/leg_lengthen.png}
    % \includegraphics[width=0.23\linewidth]{OllieFigs/tail_shorten.png}
    % \includegraphics[width=0.23\linewidth]{OllieFigs/tail_puff.png}
    \includegraphics[width=.95\linewidth]{OllieFigs/all_shapevar.png}
    \caption{\textbf{Effect of varying SMBLD scale parameters}. 
    \emph{From left to right}: 
    Mean SMBLD model, 
    25\% leg elongation,
    50\% tail elongation,
    50\% ear elongation.}
    \label{fig:shape_variation}
\end{figure*}

\subsection{Building a 3D shape prior via model fitting}

Another method for improving the generalizability of the SMAL model is to improve the 3D shape prior. Such priors are typically used to ensure shape deformation remain within a realistic and anatomically plausible range. Due to the limited diversity of scans used to build the SMAL model, while the shape prior does enforce realism among deformations, it does not allow for a wide enough range to cover the set of dogs in our dataset.

We improve the quality of the prior (and learn a prior over our new scale parameters) by fitting to a set of $13$ artist-designed 3D dog meshes, which are more varied than the original set. We apply an energy minimization scheme which aligns the SMAL vertices to each scan, under smoothing regularizers. Further details left to the supplementary.

% Way more here, and include exampels

\section{End-to-end dog reconstruction from monocular images} 

We now consider the task of reconstructing a 3D dog mesh from a monocular image. We achieve this by training an end-to-end convolutional network that predicts a set of SMBLD model and perspective camera parameters. In particular, we train our network to predict pose $\pose$ and shape $\shape$ SMBLD parameters together with translation $\trans$ and focal length $f$ for a perspective camera. A complete overview of the proposed system is shown in Figure~\ref{fig:sys_overview_train_sup}.

\subsection{Model architecture}

%extended with convolutional layer and an fully-connected layer 
Our network architecture is inspired by the model of 3D-Safari~\cite{Zuffi19Safari}. Given an input image cropped to (224, 224), we apply a Resnet-50~\cite{he2016deep} backbone network to encode a 1024-dimensional feature map. These features are passed through various linear prediction heads to produce the required parameters. The pose, translation and camera prediction modules follow the design of 3D-Safari, but we describe the differences in our shape module.

\ss{Pose, translation and camera prediction.}
These modules are independent multi-layer perceptrons which map the above features to the various parameter types. As with 3D-Safari we use two linear layers to map to a set of $35 \times 3$ 3D pose parameters (three parameters for each joint in the SMBLD kinematic tree) given in Rodrigues form. We use independent heads to predict camera frame translation $\trans_{x,y}$ and depth $\trans_{z}$ independently. We also predict the focal length of the perspective camera similarly to 3D-Safari.

\ss{Shape and scale prediction.}

Unlike 3D-Safari, we design our network to predict the set of shape parameters (including scale) rather than vertex offsets. We observe improvement by handling the standard 20 blend-shape parameters and our new scale parameters in separate linear prediction heads. We retrieve the scale parameters by $\scale = \exp{x}$ where $x$ are the network predictions, as we find predicting log scale helps stabilise early training.

\subsection{Training losses}

A common approach for training such an end-to-end system would be to supervise the prediction of $(\pose, \shape, \trans, \f)$ with 3D ground truth annotations~\cite{kolotouros19learning,kanazawa18end-to-end,pavlakos18learning}. However, building a suitable 3D annotation dataset would require an experienced graphics artist to design an accurate ground truth mesh for each of 20,520 StanfordExtra dog images, a prohibitive expense.


We instead develop a method that instead relies on \emph{weak 2D supervision} to guide network training. In particular, we rely on only 2D keypoints and silhouette segmentations, are significantly cheaper to obtain.

The rest of this section describes the set of losses used to supervise the network at train time.

\ss{Joint reprojection.}
The most important loss to promote accurate limb positioning is the joint reprojection loss $\L{joints}$ which compares the projected model joints $\pi(F_{J}(\pose, \shape), \trans, \f)$ to the ground truth annotations $\hat{X}$. Given the parameters predicted by the network, we apply the SMBLD model to transform the pose and shape parameters into a set of 3D joint positions $J \in \RR{35}{3}$, and project them to the image plane using translation and camera parameters. The joint loss $L_{joints}$ is given by the $\ell_2$ error between the ground truth and projected joints:

\begin{equation}
\L{joints}(\pose, \shape, \trans, \f; \hat{X}) = \lVert \hat{X} - \pi(F_{J}(\pose, \shape), \trans, \f) \rVert_{2}
\end{equation}

Note that many of our training images exhibit significant occlusion, so $\hat{X}$ contains many invisible joints. We handle this by masking $\L{joints}$ to prevent invisible joints contributing to the loss.

\ss{Silhouette loss.}
The silhouette loss $\L{sil}$ is used to promote shape alignment between the SMBLD dog mesh and the input dog. In order to compute the silhouette loss, we define a rendering function $R: (\verts, \trans, \f) \mapsto S$ which projects the SMBLD mesh to produce a binary segmentation mask. In order to allow derivatives to be propagated through $R$, we implement $R$ using the differentiable Neural Mesh Renderer~\cite{kato2018renderer}. The loss is computed as the $\ell_2$ difference between a projected silhouette and the ground truth mask $\hat{S}$:

\begin{equation}
\L{sil}(\pose, \shape, \trans, \f; \hat{S}) = \lVert \hat{S} - R\bigl(F_{V}(\pose, \shape), \trans, \f \bigr) \rVert_{2}
\end{equation}


\ss{Priors.}
In the absence of 3D ground truth training data, we rely on priors obtained from artist graphics models to encourage realism in the network predictions. We model both pose and shape using a multivariate Gaussian prior, consisting of means $\mu_{\pose},\mu_{\shape}$ and covariance matrices $\Sigma_{\pose},\Sigma_{\shape}$. The loss is given as the log likelihood of a given shape or pose vector under these distributions, which corresponds to the Mahalanobis distance between the predicted parameters and their corresponding means:
\begin{align}
    \L{pose}(\pose; \mu_{\pose}, \Sigma_{\pose}) &= (\pose - \mu_{\pose})^T \Sigma_{\pose}^{-1} (\pose - \mu_{\pose})\\
    \L{shape}(\shape; \mu_{\shape}, \Sigma_{\shape}) &= (\shape - \mu_{\shape})^T \Sigma_{\shape}^{-1} (\shape - \mu_{\shape})
\end{align}
Unlike previous work, we find there is no need to use a loss to penalize pose parameters if they exceed manually specified joint angle limits. We suspect our network learns this regularization naturally because of our large dataset.

%our network this is a positive side-effect of training a network on a large dataset rather than optimizing independently to single images, as the network can learn natural regularisation that discourages infeasible joint configurations.

%this is since the network is able to use the plentiful training examples to learn its own prior.

\subsection{Learning a multi-modal shape prior.}

% Using a unimodal prior tends to result in predictions which look relatively similar in shape. To promote diversity among predicted 3D dog shapes, our method extends the formulation above to incorporate a mixture of Gaussians prior. We represent the mixture as a set of $M$ Gaussians, whose means are initialized by drawing samples from our existing prior:

% \begin{align}
%     \mu_{\shape}^{m} &\sim N(\mu_{\shape}, \Sigma_{\shape}) \\
%     \Sigma_{\shape}^{m} &:= \Sigma_{\shape}
% \end{align}

% We assign each training image $i$ with a set of mixture weights $\{w_{i}^{1}, \dots w_{i}^{M}\}$, where initially $w_{i}^{m} := \frac{1}{M}.$

% We can then apply the following mixture shape loss:

% \begin{equation}
%     L_{mixture}=\sum_{m=1}^M w_{i}^{m}L_{shape}(\shape_{i}, \mu_{\shape}^{m}, \Sigma_{\shape}^{m})
% \end{equation}

% In order to allow our mixture prior to learn ``in-the-loop" from the available training data, we apply expectation maximization every $k$ epochs during training. This step recomputes the means and variances for each mixture component based on the observed shapes in the training set, and updates the per-image mixture weights:

% \begin{align}
%     \mu_{\shape}^{m} :=& \mathrm{E}_{i}[\beta_{i}W_{i}^{m}]\\
%     \Sigma_{\shape}^{m} :=& \mathrm{Cov}_{i}[\beta_{i}W_{i}^{m}, \beta_{i}W_{i}^{m}]\\
%     w_{i}^{m} :=& \frac{L_{shape}(\shape_{i}, \mu_{\shape}^{m}, \Sigma_{\shape}^{m})}{\sum_{m'}^{M}L_{shape}(\shape_{i}, \mu_{\shape}^{m'}, \Sigma_{\shape}^{m'})}
% \end{align}


% \section{Attempt 2}

The previous section introduced a unimodal, multivariate Gaussian shape prior, based on mean $\mu_{\shape}$ and covariance matrix $\Sigma_{\shape}$. However, we find enforcing this prior throughout training tends to result in predictions which appear similar in 3D shape, even when tested on dog images of different breeds. We propose to improve diversity among predicted 3D dog shapes by extending the above formulation to a Mixture of $M$ Gaussians prior.  
The mixture shape loss is then given as:
\begin{align}
    \L{mixture}(\shape_{i}; \mu_{\shape}, \Sigma_{\shape}, \Pi_{\shape})
    % =&
    % \sum_{m=1}^M
    % \Pi_{\shape}^m
    % (\shape_{i} - \mu_{\shape}^{m})^{T} \inv{\Sigma_{\shape}^{m}} (\shape_{i} - \mu_{\shape}^{m})
    % \\
    =&
    \sum_{m=1}^M \Pi_{\shape}^{m}\L{shape}(\shape_{i}; \mu_{\shape}^{m}, \Sigma_{\shape}^{m})
\end{align}
Where $\mu_{\shape}^{m}$, $\Sigma_{\shape}^{m}$ and $\Pi_{\shape}^{m}$ 
are the mean, covariance and mixture weight respectively for Gaussian component 
$m$. For each component the mean is sampled from our existing unimodal prior and the covariance is set equal to the unimodal prior i.e. $\Sigma_{\shape}^{m} := \Sigma_{\shape}$. All mixture weights are initially set to $\frac{1}{M}$.

Each training image $i$ is assigned a set of latent variables $\{w_{i}^{1}, \dots w_{i}^{M}\}$ encoding the probability of the dog shape in image~$i$ being generated by component~$m$. 

\subsection{Expectation Maximization in the loop}

As previously discussed, our initial shape prior is obtained from artist data which we find is unrepresentative of the diverse shapes present in our real dog dataset. We address this by proposing to recover the latent variables $w_{i}^{m}$ and parameters ($\mu_{\shape}^{m}$, $\Sigma_{\shape}^{m}$ and $\Pi_{\shape}^{m}$) of our 3D shape prior by learning from monocular images of in-the-wild dogs and their 2D training labels in our training dataset.

We achieve this using Expectation Maximization (EM), which regularly updates the means and variances for each mixture component and per-image mixture weights based on the observed shapes in the training set. While training our 3D reconstruction network, we progressively update our shape mixture model with an alternating `E' step and `M' step described below:

\subsubsection{The `E' Step.}
The `E' step computes the expected value of the latent variables~$w_{i}^{m}$ 
assuming fixed $(\mu_{\shape}^{m}, \Sigma_{\shape}^{m}, \Pi_{\shape}^{m})$ for all $i \in \{1,\dots,N\}, m \in \{1,\dots,M\}$.

The update equation for an image $i$ with latest shape prediction $\shape_{i}$ 
and cluster $m$ with parameters $(\mu_{\shape}^{m}, \Sigma_{\shape}^{m}, \Pi_{\shape}^{m})$ 
is given as:
% distance between the latest shape prediction $\shape_{i}$ and the cluster $(\mu_{\shape}^{m}, \Sigma_{\shape}^{m})$

% \begin{align}
%     w_{i}^{m} 
%     :=& 
%     \frac{
%         (\shape_{i} - \mu_{\shape}^{m})^{T} \inv{\Sigma_{\shape}^{m}} (\shape_{i} - \mu_{\shape}^{m})
%     }
%     {
%         \sum_{m'}^{M}
%         (\shape_{i} - \mu_{\shape}^{m'})^{T} \inv{\Sigma_{\shape}^{m'}} (\shape_{i} - \mu_{\shape}^{m'})
%     }
%     \\
%     :=&
%     \frac{
%         L_{shape}(\shape_{i}, \mu_{\shape}^{m}, \Sigma_{\shape}^{m})
%     }
%     {
%         \sum_{m'}^{M}L_{shape}(\shape_{i}, \mu_{\shape}^{m'}, \Sigma_{\shape}^{m'})
%     } 
% \end{align}

\begin{align}
    w_{i}^{m} 
    :=& 
    \frac{
        \mathcal{N}(\shape_{i} | \mu_{\shape}^{m},\Sigma_{\shape}^{m})\Pi_{\shape}^{m}
    }
    {
        \sum_{m'}^{M}
        \mathcal{N}(\shape_{i} | \mu_{\shape}^{m'},\Sigma_{\shape}^{m'})\Pi_{\shape}^{m'}
    }
    % \\
    % :=&
    % \frac{
    %     L_{shape}(\shape_{i}, \mu_{\shape}^{m}, \Sigma_{\shape}^{m})
    % }
    % {
    %     \sum_{m'}^{M}L_{shape}(\shape_{i}, \mu_{\shape}^{m'}, \Sigma_{\shape}^{m'})
    % } 
\end{align}



\subsubsection{The `M' Step.}
The `M' step computes new values for $(\mu_{\shape}^{m}, \Sigma_{\shape}^{m}, \Pi_{\shape}^{m})$, assuming fixed $w_{i}^{m}$ for all $i \in \{1,\dots,N\}, m \in \{1,\dots,M\}$.

The update equations are given as follows:

% \begin{align}
%     \mu_{\shape}^{m} :=& 
%     \frac{
%         \sum_{i}w_{i}^{m}\shape_{i}
%     }
%     {
%         \sum_{i}w_{i}^{m}
%     }
%     \\
%     \Sigma_{\shape}^{m} :=& 
%     \frac{
%         \sum_{i}w_{i}^{m}
%         (\shape_{i} - \Sigma_{\shape}^{m})
%         (\shape_{i} - \Sigma_{\shape}^{m})^{T}
%     }
%     {
%         \sum_{i}w_{i}^{m}
%     }
%     \\
%     \Pi_{\shape}^{m} :=& 
%     \frac{1}{N}\sum_{i}{w_{i}^{m}}.
% \end{align}


\begin{equation}
    \mu_{\shape}^{m} := 
    \frac{
        \sum_{i}w_{i}^{m}\shape_{i}
    }
    {
        \sum_{i}w_{i}^{m}
    }
    \quad
    \Sigma_{\shape}^{m} :=
    \frac{
        \sum_{i}w_{i}^{m}
        (\shape_{i} - \Sigma_{\shape}^{m})
        (\shape_{i} - \Sigma_{\shape}^{m})^{T}
    }
    {
        \sum_{i}w_{i}^{m}
    }
    \quad
    \Pi_{\shape}^{m} :=
    \frac{1}{N}\sum_{i}{w_{i}^{m}}
\end{equation}

\section{Building StanfordExtra: a new large-scale dog keypoint dataset}

\begin{figure*}[h]
    \centering
    \includegraphics[height=0.1775\textheight]{OllieFigs/collage_wide.png}
    \includegraphics[height=0.1775\textheight]{OllieFigs/heatmap.png}
    \caption{\textbf{StanfordExtra example images}. \emph{Left}: outlined segmentations and labelled keypoints for 24 representative images. \emph{Right}: heatmap of deviation of worker submitted results from mean for each submission.}
    \label{fig:dataset}
\end{figure*}

In order to evaluate our method, we introduce \emph{StanfordExtra}: a new large-scale dataset with annotated 2D keypoints and binary segmentation masks for dogs. We opted to take source images from the existing Stanford Dog Dataset~\cite{StanfordDogs}, which consists of 20,580 dog images taken ``in the wild" and covers 120 dog breeds. The dataset contains vast shape and pose variation between dogs, as well as nuisance factors such as self/environmental occlusion, interaction with humans/other animals and partial views. Figure~\ref{fig:dataset} (left) shows samples from the new dataset.

We used Amazon Mechanical Turk to collect a binary silhouette mask and 20 keypoints per image: 3 per leg (knee, ankle, toe), 2 per ear (base, tip), 2 per tail (base, tip), 2 per face (nose and jaw). We can approximate the difficulty of the dataset by analysing the variance between 3 annotators at both the joint labelling and silhouette task. Figure~\ref{fig:dataset} (right) illustrates typical per-joint variance in joint labelling. Further details of the data curation procedure are left to the supplementary materials. 


\section{Experiments}

% As with other methods, we observe that introducing a silhouette term into the network loss before the model's pose has been somewhat solved can result in unsatisfactory local minima. We overcome this by using a pre-training stage with the following loss terms:


In this section we compare our method to competitive baselines. We begin by describing our new large-scale dataset of annotated dog images, followed by a quantitative and qualitative evaluation.

\subsection{Evaluation protocol}

Our evaluation is based on our new StanfordExtra dataset. In line with other methods which tackle ``in-the-wild'' 3D reconstruction of articulated subjects~\cite{kolotouros19learning,kolotouros19convolutional}, we filter images from the original dataset of 20,580 for which the majority of dog keypoints are invisible. We consider these images unsuitable for our full-body dog reconstruction task. We also remove images for which the consistency in keypoint/silhouette segmentations between the 3 annotators is below a set threshold. This leaves us with 8,476 images which we divide per-breed into an 80\%/20\% train and test split.

We consider two primary evaluation metrics. IoU is the intersection-over-union of the projected model silhouette compared to the ground truth annotation and indicates the quality of the reconstructed 3D shape. Percentage of Correct Keypoints (PCK) computes the percentage of joints which are within a normalized distance (based on square root of 2D silhouette area) to the ground truth locations, and evaluates the quality of reconstructed 3D pose. We also produce PCK results on various joint groups (legs, tail, ears, face) to compare the reconstruction accuracy for different parts of the dog model.

\subsection{Training procedure}

We train our model in two stages. The first omits the silhouette loss which we find can lead the network to unsatisfactory local minima if applied too early. With the silhouette loss turned off, we find it satisfactory to use the simple unimodal prior (and without EM) for this preliminary stage since there is no loss to specifically encourage a strong shape alignment. After this, we introduce the silhouette loss, the mixture prior and begin applying the expectation maximization updates over $M=10$ clusters. We train the first stage for 250 epochs, the second stage for 150 and apply the EM step every 50 epochs. All losses are weighted, as described in the supplementary. The entire training procedure takes 96 hours on a single P100 GPU.

\subsection{Comparison to baselines}

We first compare our method to various baseline methods. SMAL~\cite{zuffi2017menagerie} is an approach which fits the 3D SMAL model using per-image energy minimization. Creatures Great and SMAL (CGAS)~\cite{biggs2018creatures} is a three-stage method, which employs a joint predictor on silhouette renderings from synthetic 3D dogs, applies a genetic algorithm to clean predictions, and finally applies the SMAL optimizer to produce the 3D mesh.

At test-time both SMAL and CGAS rely on manually-provided segementation masks, and SMAL also relies on hand-clicked keypoints. In order to produce a fair comparison, we produce a set of \emph{predicted} keypoints for StanfordExtra by training the Stacked Hourglass Network~\cite{newell2016stacked} with 8 stacks and 1 block, and \emph{predicted} segmentation masks using DeepLab v3+~\cite{deeplabv3plus}. The Stacked Hourglass Network achieves 71.4\% PCK score, DeepLab v3+ achieves 83.4\% IoU score and the CGAS joint predictor achieves 41.8\% PCK score. 

%All methods are trained from scratch and evaluated on our Stanford Dog validation set.

Table~\ref{tab:baselines} and Figure~\ref{fig:comparison_sup} show the comparison against competitive methods. For full examination, we additionally provide results for SMAL and CGAS in the scenario that ground-truth keypoints and/or segmentations are available at test time. 

The results show our end-to-end method outperforms the competitors when they are provided with predicted keypoints/segmentations (white rows). Our method therefore achieves a new state-of-the-art on this 3D reconstruction task. In addition, we show our method achieves improved average IoU/PCK scores than competitive methods, even when they are provided ground truth annotations at test time (grey rows). We also demonstrate wider applicability of two contributions from our work (scale parameters and improved prior) by showing improved performance of the SMAL method when these are incorporated. Finally, our model's test-time speed is significantly faster than the competitors as it does not require an optimizer.

\begin{table}[]
{
    \small
    \centering
    \begin{tabular}{@{}lcccccccc@{}}
    \toprule
    \multicolumn{1}{l}{Method} & 
    \multicolumn{1}{c}{Kps} & 
    \multicolumn{1}{c}{Seg} & 
    \multicolumn{1}{c}{IoU} & 
    \multicolumn{5}{c}{PCK} \\
    \multicolumn{4}{c}{} &
    \multicolumn{1}{c}{Avg} &
    \multicolumn{1}{c}{Legs} &
    \multicolumn{1}{c}{Tail} &
    \multicolumn{1}{c}{Ears} &
    \multicolumn{1}{c}{Face} \\
    \midrule
    SMAL~\cite{DBLP:journals/corr/ZuffiKJB16} & Pred & Pred & 67.9 & 67.1 & 65.7 & 79.5 & 54.9 & 87.4  \\
    SMAL & GT & GT &  69.2 & 72.6 & 69.9 & \textbf{92.0} & 58.6 & \textbf{96.9} \\
    SMAL & GT & Pred & 68.6 & 72.6 & 70.2 & 91.5 & 58.1 & \textbf{96.9} \\ 
    SMAL & Pred & GT & 68.5 & 67.4 & 66.0 & 79.9 & 55.0 & 88.2 \\ 
    % \rowcolor[gray]{.9} SMAL & GT & GT &  69.2 & 72.6 & 69.9 & \textbf{92.0} & 58.6 & \textbf{96.9} \\
    % \rowcolor[gray]{.9} SMAL & GT & Pred & 68.6 & 72.6 & 70.2 & 91.5 & 58.1 & \textbf{96.9} \\ 
    % \rowcolor[gray]{.9} SMAL & Pred & GT & 68.5 & 67.4 & 66.0 & 79.9 & 55.0 & 88.2 \\ 
    \hline
    CGAS~\cite{biggs2018creatures} & CGAS & Pred & 62.4 & 43.7 & 46.5 & 64.1 & 36.5 & 21.4  \\
    % \rowcolor[gray]{.9} CGAS & CGAS & GT & 63.1 & 43.6 & 46.3 & 64.2 & 36.3 & 21.6 \\
    CGAS & CGAS & GT & 63.1 & 43.6 & 46.3 & 64.2 & 36.3 & 21.6 \\
    \hline
    % \rowcolor[gray]{.9} SMAL + scaling & Pred & Pred & 69.3 & 69.6 & 69.4 & 79.3 & 56.5 & 87.6 \\
    % \rowcolor[gray]{.9} SMAL + scaling + new prior & Pred & Pred & 70.7 & 71.6 & 71.5 & 80.7 & 59.3 & 88.0 \\
    SMAL + scaling & Pred & Pred & 69.3 & 69.6 & 69.4 & 79.3 & 56.5 & 87.6 \\
    SMAL + scaling + new prior & Pred & Pred & 70.7 & 71.6 & 71.5 & 80.7 & 59.3 & 88.0 \\
    \hline
    \textbf{Ours} & --- & --- & \textbf{73.6} & \textbf{75.7} & \textbf{75.0} & 77.6 & \textbf{69.9} & 90.0 \\
    \bottomrule 
    \end{tabular}
    \vspace{1em}
    \caption{\label{tab:baselines}\textbf{Baseline comparisons.} Both PCK and silhouette IOU scores are shown for SOTA methods under varying conditions. A combination of both ground truth (GT) and predicted (Pred) keypoints/segmentations using hourglass network and deeplab respectively. For the CGAS method we also test using their keypoint predictor (CGAS). The addition of scaling and new prior are shown to improve the original SMAL method.}
}
\end{table}
\newcolumntype{?}{!{\vrule width 1pt}}

% \newcommand\sfac{0.082}
\newcommand\sfac{0.11}
\newcommand\spacercomp{4mm}
\begin{figure*}[ht!]
    \centering
    \setkeys{Gin}{width=\linewidth}
    \renewcommand\tabularxcolumn[1]{>{\Centering}m{\sfac\linewidth}} % set all columns to be centered v & hwise, with a fixed length
    % \begin{tabularx}{\textwidth}{c*{5}{X}@{\hspace{\spacercomp}}*{5}{X}c}
    \begin{tabularx}{\textwidth}{c*{5}{X}}
       \textbf{Ours} &
      \includegraphics{ours_sup/n02086646-Blenheim_spaniel/orig/n02086646_1476.jpg} &
      \includegraphics{ours_sup/n02086646-Blenheim_spaniel/fit/n02086646_1476.jpg} &
      \includegraphics{ours_sup/n02086646-Blenheim_spaniel/model/n02086646_1476_crop.jpg} &
      \includegraphics{ours_sup/n02086646-Blenheim_spaniel/joints/n02086646_1476.jpg} &
      \includegraphics{ours_sup/n02086646-Blenheim_spaniel/segs/n02086646_1476.jpg} \\
      

      3D-M &
      \includegraphics{comp_sup/smal/n02086646-Blenheim_spaniel/orig/n02086646_1476.jpg} &
      \includegraphics{comp_sup/smal/n02086646-Blenheim_spaniel/fit/n02086646_1476.jpg} &
      \includegraphics{comp_sup/smal/n02086646-Blenheim_spaniel/model/n02086646_1476_fixed_crop.jpg} &
      \includegraphics{comp_sup/smal/n02086646-Blenheim_spaniel/joints/n02086646_1476.jpg} &
      \includegraphics{comp_sup/smal/n02086646-Blenheim_spaniel/segs/n02086646_1476.jpg} \\
      


      CGAS &
      \includegraphics{comp_sup/cgas/n02086646-Blenheim_spaniel/orig/n02086646_1476.jpg} &
      \includegraphics{comp_sup/cgas/n02086646-Blenheim_spaniel/fit/n02086646_1476.jpg} &
      \includegraphics{comp_sup/cgas/n02086646-Blenheim_spaniel/model/n02086646_1476_fixed_crop.jpg} &
      \includegraphics{comp_sup/cgas/n02086646-Blenheim_spaniel/joints/n02086646_1476.jpg} &
      \includegraphics{comp_sup/cgas/n02086646-Blenheim_spaniel/segs/n02086646_1476.jpg} \\
      %\hspace{\spacercomp}
     
      \textbf{Ours} &
      \includegraphics{ours_sup/n02087394-Rhodesian_ridgeback/orig/n02087394_831.jpg} &
      \includegraphics{ours_sup/n02087394-Rhodesian_ridgeback/fit/n02087394_831.jpg} &
      \includegraphics{ours_sup/n02087394-Rhodesian_ridgeback/model/n02087394_831_crop.jpg} &
      \includegraphics{ours_sup/n02087394-Rhodesian_ridgeback/joints/n02087394_831.jpg} &
      \includegraphics{ours_sup/n02087394-Rhodesian_ridgeback/segs/n02087394_831.jpg} \\

      3D-M &
      %\hspace{\spacercomp}
      \includegraphics{comp_sup/smal/n02087394-Rhodesian_ridgeback/orig/n02087394_831.jpg} &
      \includegraphics{comp_sup/smal/n02087394-Rhodesian_ridgeback/fit/n02087394_831.jpg} &
      \includegraphics{comp_sup/smal/n02087394-Rhodesian_ridgeback/model/n02087394_831_fixed_crop.jpg} &
      \includegraphics{comp_sup/smal/n02087394-Rhodesian_ridgeback/joints/n02087394_831.jpg} &
      \includegraphics{comp_sup/smal/n02087394-Rhodesian_ridgeback/segs/n02087394_831.jpg} \\

      CGAS &
      \includegraphics{comp_sup/cgas/n02087394-Rhodesian_ridgeback/orig/n02087394_831.jpg} &
      \includegraphics{comp_sup/cgas/n02087394-Rhodesian_ridgeback/fit/n02087394_831.jpg} &
      \includegraphics{comp_sup/cgas/n02087394-Rhodesian_ridgeback/model/n02087394_831_fixed_crop.jpg} &
      \includegraphics{comp_sup/cgas/n02087394-Rhodesian_ridgeback/joints/n02087394_831.jpg} &
      \includegraphics{comp_sup/cgas/n02087394-Rhodesian_ridgeback/segs/n02087394_831.jpg} \\
      
      \textbf{Ours} &
      \includegraphics{ours_sup/n02091134-whippet/orig/n02091134_16201.jpg} &
      \includegraphics{ours_sup/n02091134-whippet/fit/n02091134_16201.jpg} &
      \includegraphics{ours_sup/n02091134-whippet/model/n02091134_16201_crop.jpg} &
      \includegraphics{ours_sup/n02091134-whippet/joints/n02091134_16201.jpg} &
      \includegraphics{ours_sup/n02091134-whippet/segs/n02091134_16201.jpg} \\


      3D-M & 
      \includegraphics{comp_sup/smal/n02091134-whippet/orig/n02091134_16201.jpg} &
      \includegraphics{comp_sup/smal/n02091134-whippet/fit/n02091134_16201.jpg} &
      \includegraphics{comp_sup/smal/n02091134-whippet/model/n02091134_16201_fixed_crop.jpg} &
      \includegraphics{comp_sup/smal/n02091134-whippet/joints/n02091134_16201.jpg} &
      \includegraphics{comp_sup/smal/n02091134-whippet/segs/n02091134_16201.jpg} \\
      
      
      CGAS &
      \includegraphics{comp_sup/cgas/n02091134-whippet/orig/n02091134_16201.jpg} &
      \includegraphics{comp_sup/cgas/n02091134-whippet/fit/n02091134_16201.jpg} &
      \includegraphics{comp_sup/cgas/n02091134-whippet/model/n02091134_16201_fixed_crop.jpg} &
      \includegraphics{comp_sup/cgas/n02091134-whippet/joints/n02091134_16201.jpg} &
      \includegraphics{comp_sup/cgas/n02091134-whippet/segs/n02091134_16201.jpg} \\

      \textbf{Ours} &
      \includegraphics{ours_sup/n02089078-black-and-tan_coonhound/orig/n02089078_877.jpg} &
      \includegraphics{ours_sup/n02089078-black-and-tan_coonhound/fit/n02089078_877.jpg} &
      \includegraphics{ours_sup/n02089078-black-and-tan_coonhound/model/n02089078_877_crop.jpg} &
      \includegraphics{ours_sup/n02089078-black-and-tan_coonhound/joints/n02089078_877.jpg} &
      \includegraphics{ours_sup/n02089078-black-and-tan_coonhound/segs/n02089078_877.jpg} \\

      3D-M & 
      \includegraphics{comp_sup/smal/n02089078-black-and-tan_coonhound/orig/n02089078_877.jpg}&
      \includegraphics{comp_sup/smal/n02089078-black-and-tan_coonhound/fit/n02089078_877.jpg}&
      \includegraphics{comp_sup/smal/n02089078-black-and-tan_coonhound/model/n02089078_877_fixed_crop.jpg}&
      \includegraphics{comp_sup/smal/n02089078-black-and-tan_coonhound/joints/n02089078_877.jpg}&
      \includegraphics{comp_sup/smal/n02089078-black-and-tan_coonhound/segs/n02089078_877.jpg} \\
      
      CGAS &
      \includegraphics{comp_sup/cgas/n02089078-black-and-tan_coonhound/orig/n02089078_877.jpg}&
      \includegraphics{comp_sup/cgas/n02089078-black-and-tan_coonhound/fit/n02089078_877.jpg}&
      \includegraphics{comp_sup/cgas/n02089078-black-and-tan_coonhound/model/n02089078_877_fixed_crop.jpg}&
      \includegraphics{comp_sup/cgas/n02089078-black-and-tan_coonhound/joints/n02089078_877.jpg}&
      \includegraphics{comp_sup/cgas/n02089078-black-and-tan_coonhound/segs/n02089078_877.jpg} \\

      % \textbf{Ours} &
      % \includegraphics{ours_sup/n02087394-Rhodesian_ridgeback/orig/n02087394_831.jpg} &
      % \includegraphics{ours_sup/n02087394-Rhodesian_ridgeback/fit/n02087394_831.jpg} &
      % \includegraphics{ours_sup/n02087394-Rhodesian_ridgeback/model/n02087394_831.jpg} &
      % \includegraphics{ours_sup/n02087394-Rhodesian_ridgeback/joints/n02087394_831.jpg} &
      % \includegraphics{ours_sup/n02087394-Rhodesian_ridgeback/segs/n02087394_831.jpg} &
      % %\hspace{\spacercomp}


      % SMAL &
      % \includegraphics{comp_sup/smal/n02087394-Rhodesian_ridgeback/orig/n02087394_831.jpg} &
      % \includegraphics{comp_sup/smal/n02087394-Rhodesian_ridgeback/fit/n02087394_831.jpg} &
      % \includegraphics{comp_sup/smal/n02087394-Rhodesian_ridgeback/model/n02087394_831.jpg} &
      % \includegraphics{comp_sup/smal/n02087394-Rhodesian_ridgeback/joints/n02087394_831.jpg} &
      % \includegraphics{comp_sup/smal/n02087394-Rhodesian_ridgeback/segs/n02087394_831.jpg} &
      % %\hspace{\spacercomp}
      % \includegraphics{comp_sup/smal/n02089078-black-and-tan_coonhound/orig/n02089078_877.jpg} &
      % \includegraphics{comp_sup/smal/n02089078-black-and-tan_coonhound/fit/n02089078_877.jpg} &
      % \includegraphics{comp_sup/smal/n02089078-black-and-tan_coonhound/model/n02089078_877.jpg} &
      % \includegraphics{comp_sup/smal/n02089078-black-and-tan_coonhound/joints/n02089078_877.jpg} &
      % \includegraphics{comp_sup/smal/n02089078-black-and-tan_coonhound/segs/n02089078_877.jpg} \\

      % CGAS &
      % \includegraphics{comp_sup/cgas/n02087394-Rhodesian_ridgeback/orig/n02087394_831.jpg} &
      % \includegraphics{comp_sup/cgas/n02087394-Rhodesian_ridgeback/fit/n02087394_831.jpg} &
      % \includegraphics{comp_sup/cgas/n02087394-Rhodesian_ridgeback/model/n02087394_831.jpg} &
      % \includegraphics{comp_sup/cgas/n02087394-Rhodesian_ridgeback/joints/n02087394_831.jpg} &
      % \includegraphics{comp_sup/cgas/n02087394-Rhodesian_ridgeback/segs/n02087394_831.jpg} &
      % %\hspace{\spacercomp}
      % \includegraphics{comp_sup/cgas/n02089078-black-and-tan_coonhound/orig/n02089078_877.jpg} &
      % \includegraphics{comp_sup/cgas/n02089078-black-and-tan_coonhound/fit/n02089078_877.jpg} &
      % \includegraphics{comp_sup/cgas/n02089078-black-and-tan_coonhound/model/n02089078_877.jpg} &
      % \includegraphics{comp_sup/cgas/n02089078-black-and-tan_coonhound/joints/n02089078_877.jpg} &
      % \includegraphics{comp_sup/cgas/n02089078-black-and-tan_coonhound/segs/n02089078_877.jpg} \\

      % \textbf{Ours} &
      % \includegraphics{ours_sup/n02085936-Maltese_dog/orig/n02085936_3313.jpg} &
      % \includegraphics{ours_sup/n02085936-Maltese_dog/fit/n02085936_3313.jpg} &
      % \includegraphics{ours_sup/n02085936-Maltese_dog/model/n02085936_3313.jpg} &
      % \includegraphics{ours_sup/n02085936-Maltese_dog/joints/n02085936_3313.jpg} &
      % \includegraphics{ours_sup/n02085936-Maltese_dog/segs/n02085936_3313.jpg} &
      % %\hspace{\spacercomp}
      % \includegraphics{ours_sup/n02088238-basset/orig/n02088238_10054.jpg} &
      % \includegraphics{ours_sup/n02088238-basset/fit/n02088238_10054.jpg} &
      % \includegraphics{ours_sup/n02088238-basset/model/n02088238_10054.jpg} &
      % \includegraphics{ours_sup/n02088238-basset/joints/n02088238_10054.jpg} &
      % \includegraphics{ours_sup/n02088238-basset/segs/n02088238_10054.jpg} \\

      % SMAL &
      % \includegraphics{comp_sup/smal/n02085936-Maltese_dog/orig/n02085936_3313.jpg} &
      % \includegraphics{comp_sup/smal/n02085936-Maltese_dog/fit/n02085936_3313.jpg} &
      % \includegraphics{comp_sup/smal/n02085936-Maltese_dog/model/n02085936_3313.jpg} &
      % \includegraphics{comp_sup/smal/n02085936-Maltese_dog/joints/n02085936_3313.jpg} &
      % \includegraphics{comp_sup/smal/n02085936-Maltese_dog/segs/n02085936_3313.jpg} &
      % %\hspace{\spacercomp}
      % \includegraphics{comp_sup/smal/n02088238-basset/orig/n02088238_10054.jpg} &
      % \includegraphics{comp_sup/smal/n02088238-basset/fit/n02088238_10054.jpg} &
      % \includegraphics{comp_sup/smal/n02088238-basset/model/n02088238_10054.jpg} &
      % \includegraphics{comp_sup/smal/n02088238-basset/joints/n02088238_10054.jpg} &
      % \includegraphics{comp_sup/smal/n02088238-basset/segs/n02088238_10054.jpg} \\

      % CGAS &
      % \includegraphics{comp_sup/cgas/n02085936-Maltese_dog/orig/n02085936_3313.jpg} &
      % \includegraphics{comp_sup/cgas/n02085936-Maltese_dog/fit/n02085936_3313.jpg} &
      % \includegraphics{comp_sup/cgas/n02085936-Maltese_dog/model/n02085936_3313.jpg} &
      % \includegraphics{comp_sup/cgas/n02085936-Maltese_dog/joints/n02085936_3313.jpg} &
      % \includegraphics{comp_sup/cgas/n02085936-Maltese_dog/segs/n02085936_3313.jpg} &
      % %\hspace{\spacercomp}
      % \includegraphics{comp_sup/cgas/n02088238-basset/orig/n02088238_10054.jpg} &
      % \includegraphics{comp_sup/cgas/n02088238-basset/fit/n02088238_10054.jpg} &
      % \includegraphics{comp_sup/cgas/n02088238-basset/model/n02088238_10054.jpg} &
      % \includegraphics{comp_sup/cgas/n02088238-basset/joints/n02088238_10054.jpg} &
      % \includegraphics{comp_sup/cgas/n02088238-basset/segs/n02088238_10054.jpg} \\

      % & 
      & (a) & (b) & (c) & (d) & (e) \\
      %\hspace{\spacercomp} 
      % (a) & (b) & (c) & (d) & (e) \\
    \end{tabularx}
    %
    \caption{%
    \textbf{Qualitiative comparison to SOTA.} 
    Row 1: \textbf{Ours}, 
    Row 2: 3D-M~\cite{zuffi2017menagerie}, 
    Row 3: CGAS~\cite{biggs2018creatures}. 
    (a) input image, (b) predicted 3D mesh, (c) canonical view 3D mesh, 
    (d) reprojected model joints and (e) silhouette reprojection error. 
    }
    \label{fig:comparison_sup}
\end{figure*}


%xxx: Note that CGAS does badly as it can't clean up using video
\subsection{Generalization to unseen dataset}

Table~\ref{tab:animalpose} shows an experiment to compare how well our model generalizes to a new data domain. We test our model against the SMAL~\cite{zuffi2017menagerie} method (using predicted keypoints and segmentations as above for fairness) on the recent Animal Pose dataset~\cite{animalpose}. The data preparation process is the same as for StanfordExtra and no fine-tuning was used for either method. We achieve good results in this unseen domain and still improve over the SMAL optimizer.

\begin{table}
\begin{tabular}{@{}lcccccc@{}}
\toprule
\multicolumn{1}{l}{Method} & 
\multicolumn{1}{c}{IoU} & 
\multicolumn{5}{c}{PCK} \\
\multicolumn{2}{c}{} &
\multicolumn{1}{c}{Avg} &
\multicolumn{1}{c}{Legs} &
\multicolumn{1}{c}{Tail} &
\multicolumn{1}{c}{Ears} &
\multicolumn{1}{c}{Face} \\
\midrule
SMAL~\cite{zuffi2017menagerie} & 63.6 & 69.1 & 60.9 & 83.5 & 75.0 & 93.0 \\
% \hline
\textbf{Ours} & \textbf{66.9} & \textbf{73.8} & \textbf{65.1} & \textbf{85.6} & \textbf{84.0} & \textbf{93.6} \\
\bottomrule
\multicolumn{7}{c}{} \\
\multicolumn{7}{c}{}
% \textbf{Ours} & \textbf{66.9} & \textbf{73.8} & \textbf{65.1} & \textbf{85.6} & \textbf{84.0} & \textbf{93.6} \\
% \textbf{Ours} & \textbf{66.9} & \textbf{73.8} & \textbf{65.1} & \textbf{85.6} & \textbf{84.0} & \textbf{93.6}

\end{tabular}
\caption{
    \label{tab:animalpose}
    \textbf{Animal Pose dataset~\cite{animalpose}}. Evaluation on recent Animal Pose dataset with no fine-tuning to our method nor joint/silhouette predictors used for SMAL.}
\end{table}

\begin{table}
\begin{tabular}{@{}lcccccc@{}}
\toprule
\multicolumn{1}{l}{Method} & 
\multicolumn{1}{c}{IoU} & 
\multicolumn{5}{c}{PCK} \\
\multicolumn{2}{c}{} &
\multicolumn{1}{c}{Avg} &
\multicolumn{1}{c}{Legs} &
\multicolumn{1}{c}{Tail} &
\multicolumn{1}{c}{Ears} &
\multicolumn{1}{c}{Face} \\
\midrule
\textbf{Ours} & \textbf{73.6} & \textbf{75.7} & \textbf{75.0} & \textbf{77.6} & 69.9 & 90.0 \\
$-$EM & 67.7 & 74.6 & 72.9 & 75.2 & \textbf{72.5} & 88.3 \\
$-$MoG & 68.0 & 74.9 & 74.3 & 73.3 & 70.0 & \textbf{90.2} \\ 
$-$Scale & 67.3 & 72.6 & 72.9 & 75.3 & 62.3 & 89.1 \\
\bottomrule 
\end{tabular}
\caption{\label{tab:ablation}\textbf{Ablation study.} Evaluation with the following contributions removed: (a) EM updates, (b) Mixture Shape Prior, (c) SMBLD scale parameters.}
\end{table}


\subsection{Ablation study}

We also produce a study in which we ablate individual components of our method and examine the effect on the PCK/IoU performance. We evaluate three variants: (1) \textbf{Ours w/o EM} that omits EM updates, (2) \textbf{Ours w/o MoG} which replaces our mixture shape prior with a unimodal prior, (3)~\textbf{Ours w/o Scale} which removes the scale parameters. 

The results in Table~\ref{tab:ablation} indicate that each individual component has a positive impact on the overall method performance. In particular, it can be seen that the inclusion of the EM and Mixture of Gaussians prior leads to an improvement in IoU, suggesting that the shape prior refinements steps help the model accurately fit the exact dog shape. Interestingly, we notice that adding the Mixture of Gaussians prior but omitting EM steps slightly hinders performance, perhaps due to an sub-optimal initialization for the $M$ clusters. However, we find adding EM updates to the Mixture of Gaussian model improves all metrics except the ear keypoint accuracy. We observe the error here is caused by the our shape prior learning slightly imprecise shapes for dogs with extremely ``floppy'' ears. Although there is good silhouette coverage for these regions, the fact our model has only a single articulation point per ear causes a lack of flexibility that results in occasionally misplaced ear tips for these instances. This could be improved in future work by adding additional model joints to the ear. Finally, we find the increased model flexibility afforded by the SMBLD scale parameters have a positive effect on IoU/PCK scores. 


% We are able to use the dataset to examine which dog parts are the most challenging to position. 
% \input{eccv2020kit/fig_jointspreads}
% \anote{TODO: PCK tables and errors visualized on 3D dog.}
% \paragraph{Analysis over breeds}
% A significant benefit of our dog dataset is that the supplied breed labels allows for reconstruction performance to be evaluated over particular breeds. \anote{Figure} ranks the breeds by error.
% \begin{table}[]
\small
\centering
\begin{tabular}{@{}lrrrr@{}}
\toprule
\multicolumn{1}{l}{} & 
\multicolumn{1}{c}{PCK}         & 
\multicolumn{1}{c}{IoU}         & 
\multicolumn{1}{c}{Metric 3}         & 
\multicolumn{1}{c}{Metric 4}          
\\ \midrule
\multicolumn{1}{r}{Daschund}             & 109.1                    & 99.6                     & 96.5                     & 95.5                     \\

\end{tabular}
\vspace{1em}
\caption{\label{tab:breed}\textbf{Evaluation of breeds} showing that there is a difference}
\end{table}

\subsection{Qualitative evaluation}

Figure~\ref{fig:comparison_sup} shows a range of example system outputs when tested on range of StanfordExtra and Animal Pose~\cite{animalpose} dogs with varying pose and shape and in challenging conditions. Note that only StanfordExtra is used for training.


% \input{fig_comparison}

% \input{fig_qualresults}

% \input{fig_qual_results_animal_pose}
% \section{Failure Cases}

\section{Conclusions}
This paper presents an end-to-end method for automatic, monocular 3D dog reconstruction. We achieve this using only weak 2D supervision, provided by our novel StanfordExtra dataset. Further, we show we can learn a more detailed shape prior by tuning a gaussian mixture during model training and this leads to improved reconstructions. We also show our method improves over competitive baselines, even when they are given access to ground truth data at test time.

Future work should involve tackling some failure cases of our system, for example handling multiple overlapping dogs or dealing with heavy motion blur. Other areas for research include extending our EM formulation to handle video input to take advantage of multi-view shape constraints, and transferring knowledge accumulated through training on StanfordExtra dogs to other species.

\newcolumntype{?}{!{\vrule width 1pt}}

\newcolumntype{M}[1]{>{\centering\arraybackslash}m{#1}}
\newcommand\scalefactorqual{0.085}
\newcommand\spacerqual{3mm}

\begin{figure*}[t!]
    \centering
    \setkeys{Gin}{width=\linewidth}
    %M{40pt}
    \renewcommand\tabularxcolumn[1]{>{\Centering}m{\scalefactorqual\linewidth}} % set all columns to be centered v & hwise, with a fixed width
    \begin{tabularx}{\textwidth}{m{15pt}*{5}{X} @ {\hspace{\spacerqual}}*{5}{X}}%
        %260pt = 8 rows, 
        \multirow{-2}{*}{\rotatebox[origin=c]{90}{$\overbrace{\hspace{293pt}}^{\textrm{\large StanfordExtra}}$}} &
    
        % R1
        \includegraphics{ours_sup/n02093256-Staffordshire_bullterrier/orig/n02093256_5791.jpg} &
        \includegraphics{ours_sup/n02093256-Staffordshire_bullterrier/fit/n02093256_5791.jpg} &
        \includegraphics{ours_sup/n02093256-Staffordshire_bullterrier/model/n02093256_5791_crop.jpg} &
        \includegraphics{ours_sup/n02093256-Staffordshire_bullterrier/joints/n02093256_5791.jpg} &
        \includegraphics{ours_sup/n02093256-Staffordshire_bullterrier/segs/n02093256_5791.jpg} & 
        %\hspace{\spacercomp} 
        \includegraphics{ours_sup/n02088364-beagle/orig/n02088364_2499.jpg} & 
        \includegraphics{ours_sup/n02088364-beagle/fit/n02088364_2499.jpg} & 
        \includegraphics{ours_sup/n02088364-beagle/model/n02088364_2499_crop.jpg} & 
        \includegraphics{ours_sup/n02088364-beagle/joints/n02088364_2499.jpg} & 
        \includegraphics{ours_sup/n02088364-beagle/segs/n02088364_2499.jpg} \\ 

        % R2
        &\includegraphics{ours_sup/n02107908-Appenzeller/orig/n02107908_933.jpg} & 
        \includegraphics{ours_sup/n02107908-Appenzeller/fit/n02107908_933.jpg} & 
        \includegraphics{ours_sup/n02107908-Appenzeller/model/n02107908_933.jpg} & 
        \includegraphics{ours_sup/n02107908-Appenzeller/joints/n02107908_933.jpg} & 
        \includegraphics{ours_sup/n02107908-Appenzeller/segs/n02107908_933.jpg} & 
        %\hspace{\spacercomp} 
        \includegraphics{ours_sup/n02088094-Afghan_hound/orig/n02088094_1917.jpg} & 
        \includegraphics{ours_sup/n02088094-Afghan_hound/fit/n02088094_1917.jpg} & 
        \includegraphics{ours_sup/n02088094-Afghan_hound/model/n02088094_1917.jpg} & 
        \includegraphics{ours_sup/n02088094-Afghan_hound/joints/n02088094_1917.jpg} & 
        \includegraphics{ours_sup/n02088094-Afghan_hound/segs/n02088094_1917.jpg} \\ 

        % R3
        &\includegraphics{ours_sup/n02087394-Rhodesian_ridgeback/orig/n02087394_7056.jpg} & 
        \includegraphics{ours_sup/n02087394-Rhodesian_ridgeback/fit/n02087394_7056.jpg} & 
        \includegraphics{ours_sup/n02087394-Rhodesian_ridgeback/model/n02087394_7056_crop.jpg} & 
        \includegraphics{ours_sup/n02087394-Rhodesian_ridgeback/joints/n02087394_7056.jpg} & 
        \includegraphics{ours_sup/n02087394-Rhodesian_ridgeback/segs/n02087394_7056.jpg} &
        %\hspace{\spacercomp} 
        \includegraphics{ours_sup/n02099601-golden_retriever/orig/n02099601_304.jpg} & 
        \includegraphics{ours_sup/n02099601-golden_retriever/fit/n02099601_304.jpg} & 
        \includegraphics{ours_sup/n02099601-golden_retriever/model/n02099601_304_crop.jpg} & 
        \includegraphics{ours_sup/n02099601-golden_retriever/joints/n02099601_304.jpg} & 
        \includegraphics{ours_sup/n02099601-golden_retriever/segs/n02099601_304.jpg} \\ 
        
        % R4
        &\includegraphics{ours_sup/n02085620-Chihuahua/orig/n02085620_3651.jpg} & 
        \includegraphics{ours_sup/n02085620-Chihuahua/fit/n02085620_3651.jpg} & 
        \includegraphics{ours_sup/n02085620-Chihuahua/model/n02085620_3651_crop.jpg} &
        \includegraphics{ours_sup/n02085620-Chihuahua/joints/n02085620_3651.jpg} &
        \includegraphics{ours_sup/n02085620-Chihuahua/segs/n02085620_3651.jpg} &
        %\hspace{\spacercomp} 
        % \includegraphics{ours_sup/n02091134-whippet/orig/n02091134_16201.jpg} &
        % \includegraphics{ours_sup/n02091134-whippet/fit/n02091134_16201.jpg} &
        % \includegraphics{ours_sup/n02091134-whippet/model/n02091134_16201.jpg} &
        % \includegraphics{ours_sup/n02091134-whippet/joints/n02091134_16201.jpg} &
        % \includegraphics{ours_sup/n02091134-whippet/segs/n02091134_16201.jpg} \\ 
        \includegraphics{ours_sup/n02097130-giant_schnauzer/orig/n02097130_5121.jpg} &
        \includegraphics{ours_sup/n02097130-giant_schnauzer/fit/n02097130_5121.jpg} &
        \includegraphics{ours_sup/n02097130-giant_schnauzer/model/n02097130_5121_crop.jpg} &
        \includegraphics{ours_sup/n02097130-giant_schnauzer/joints/n02097130_5121.jpg} &
        \includegraphics{ours_sup/n02097130-giant_schnauzer/segs/n02097130_5121.jpg} \\
        
        % R5
        &\includegraphics{ours_sup/n02089078-black-and-tan_coonhound/orig/n02089078_877.jpg} &
        \includegraphics{ours_sup/n02089078-black-and-tan_coonhound/fit/n02089078_877.jpg} &
        \includegraphics{ours_sup/n02089078-black-and-tan_coonhound/model/n02089078_877_crop.jpg} &
        \includegraphics{ours_sup/n02089078-black-and-tan_coonhound/joints/n02089078_877.jpg} &
        \includegraphics{ours_sup/n02089078-black-and-tan_coonhound/segs/n02089078_877.jpg} &
        %\hspace{\spacercomp} 
        \includegraphics{ours_sup/n02085620-Chihuahua/orig/n02085620_1152.jpg} &
        \includegraphics{ours_sup/n02085620-Chihuahua/fit/n02085620_1152.jpg} &
        \includegraphics{ours_sup/n02085620-Chihuahua/model/n02085620_1152_crop.jpg} &
        \includegraphics{ours_sup/n02085620-Chihuahua/joints/n02085620_1152.jpg} &
        \includegraphics{ours_sup/n02085620-Chihuahua/segs/n02085620_1152.jpg} \\ 
        
        % R6
        &\includegraphics{ours_sup/n02108000-EntleBucher/orig/n02108000_3104.jpg} &
        \includegraphics{ours_sup/n02108000-EntleBucher/fit/n02108000_3104.jpg} &
        \includegraphics{ours_sup/n02108000-EntleBucher/model/n02108000_3104_crop.jpg} &
        \includegraphics{ours_sup/n02108000-EntleBucher/joints/n02108000_3104.jpg} &
        \includegraphics{ours_sup/n02108000-EntleBucher/segs/n02108000_3104.jpg} &
        %\hspace{\spacercomp} 
        \includegraphics{ours_sup/n02105162-malinois/orig/n02105162_688.jpg} &
        \includegraphics{ours_sup/n02105162-malinois/fit/n02105162_688.jpg} &
        \includegraphics{ours_sup/n02105162-malinois/model/n02105162_688_crop.jpg} &
        \includegraphics{ours_sup/n02105162-malinois/joints/n02105162_688.jpg} &
        \includegraphics{ours_sup/n02105162-malinois/segs/n02105162_688.jpg} \\ 
        
        % R7
        &\includegraphics{ours_sup/n02102318-cocker_spaniel/orig/n02102318_11648.jpg} &
        \includegraphics{ours_sup/n02102318-cocker_spaniel/fit/n02102318_11648.jpg} &
        \includegraphics{ours_sup/n02102318-cocker_spaniel/model/n02102318_11648_crop.jpg} &
        \includegraphics{ours_sup/n02102318-cocker_spaniel/joints/n02102318_11648.jpg} &
        \includegraphics{ours_sup/n02102318-cocker_spaniel/segs/n02102318_11648.jpg} &
        %\hspace{\spacercomp}
        \includegraphics{ours_sup/n02086910-papillon/orig/n02086910_3020.jpg} &
        \includegraphics{ours_sup/n02086910-papillon/fit/n02086910_3020.jpg} &
        \includegraphics{ours_sup/n02086910-papillon/model/n02086910_3020_crop.jpg} &
        \includegraphics{ours_sup/n02086910-papillon/joints/n02086910_3020.jpg} &
        \includegraphics{ours_sup/n02086910-papillon/segs/n02086910_3020.jpg} \\
        
        % R8
        % &\includegraphics{ours_sup/n02097130-giant_schnauzer/orig/n02097130_5121.jpg} &
        % \includegraphics{ours_sup/n02097130-giant_schnauzer/fit/n02097130_5121.jpg} &
        % \includegraphics{ours_sup/n02097130-giant_schnauzer/model/n02097130_5121.jpg} &
        % \includegraphics{ours_sup/n02097130-giant_schnauzer/joints/n02097130_5121.jpg} &
        % \includegraphics{ours_sup/n02097130-giant_schnauzer/segs/n02097130_5121.jpg} &
        % \hspace{\spacercomp} 
        % \includegraphics{ours_sup/n02087394-Rhodesian_ridgeback/orig/n02087394_831.jpg} &
        % \includegraphics{ours_sup/n02087394-Rhodesian_ridgeback/fit/n02087394_831.jpg} &
        % \includegraphics{ours_sup/n02087394-Rhodesian_ridgeback/model/n02087394_831.jpg} &
        % \includegraphics{ours_sup/n02087394-Rhodesian_ridgeback/joints/n02087394_831.jpg} &
        % \includegraphics{ours_sup/n02087394-Rhodesian_ridgeback/segs/n02087394_831.jpg} \\

        % R9
        &\includegraphics{ours_sup/n02091032-Italian_greyhound/orig/n02091032_1933.jpg} &
        \includegraphics{ours_sup/n02091032-Italian_greyhound/fit/n02091032_1933.jpg} &
        \includegraphics{ours_sup/n02091032-Italian_greyhound/model/n02091032_1933_crop.jpg} &
        \includegraphics{ours_sup/n02091032-Italian_greyhound/joints/n02091032_1933.jpg} &
        \includegraphics{ours_sup/n02091032-Italian_greyhound/segs/n02091032_1933.jpg} &
        %\hspace{\spacercomp}
        \includegraphics{ours_sup/n02093647-Bedlington_terrier/orig/n02093647_3594.jpg} &
        \includegraphics{ours_sup/n02093647-Bedlington_terrier/fit/n02093647_3594.jpg} &
        \includegraphics{ours_sup/n02093647-Bedlington_terrier/model/n02093647_3594_crop.jpg} &
        \includegraphics{ours_sup/n02093647-Bedlington_terrier/joints/n02093647_3594.jpg} &
        \includegraphics{ours_sup/n02093647-Bedlington_terrier/segs/n02093647_3594.jpg} \\

        %R10
        &\includegraphics{ours_sup/n02106662-German_shepherd/orig/n02106662_13599.jpg} &
        \includegraphics{ours_sup/n02106662-German_shepherd/fit/n02106662_13599.jpg} &
        \includegraphics{ours_sup/n02106662-German_shepherd/model/n02106662_13599_crop.jpg} &
        \includegraphics{ours_sup/n02106662-German_shepherd/joints/n02106662_13599.jpg} &
        \includegraphics{ours_sup/n02106662-German_shepherd/segs/n02106662_13599.jpg} &
        %\hspace{\spacercomp} 
        \includegraphics{ours_sup/n02095314-wire-haired_fox_terrier/orig/n02095314_261.jpg} &
        \includegraphics{ours_sup/n02095314-wire-haired_fox_terrier/fit/n02095314_261.jpg} &
        \includegraphics{ours_sup/n02095314-wire-haired_fox_terrier/model/n02095314_261.jpg} &
        \includegraphics{ours_sup/n02095314-wire-haired_fox_terrier/joints/n02095314_261.jpg} &
        \includegraphics{ours_sup/n02095314-wire-haired_fox_terrier/segs/n02095314_261.jpg} \\

        % R11
        \multirow{-2.1}{*}{\rotatebox[origin=c]{90}{$\overbrace{\hspace{64pt}}^{\textrm{\large Animal Pose}}$}} &
        
        \includegraphics{ours_sup/animal_pose_fits/orig/2007_000063.jpg} &
        \includegraphics{ours_sup/animal_pose_fits/fit/2007_000063.jpg} &
        \includegraphics{ours_sup/animal_pose_fits/model/2007_000063_crop.jpg} &
        \includegraphics{ours_sup/animal_pose_fits/joints/2007_000063.jpg} &
        \includegraphics{ours_sup/animal_pose_fits/segs/2007_000063.jpg} & 
        %\hspace{\spacercomp} 
        \includegraphics{ours_sup/animal_pose_fits/orig/2007_004189.jpg} &
        \includegraphics{ours_sup/animal_pose_fits/fit/2007_004189.jpg} &
        \includegraphics{ours_sup/animal_pose_fits/model/2007_004189_crop.jpg} &
        \includegraphics{ours_sup/animal_pose_fits/joints/2007_004189.jpg} &
        \includegraphics{ours_sup/animal_pose_fits/segs/2007_004189.jpg} \\ 

        % R12
        &\includegraphics{ours_sup/animal_pose_fits/orig/2007_008222.jpg} &
        \includegraphics{ours_sup/animal_pose_fits/fit/2007_008222.jpg} &
        \includegraphics{ours_sup/animal_pose_fits/model/2007_008222_crop.jpg} &
        \includegraphics{ours_sup/animal_pose_fits/joints/2007_008222.jpg} &
        \includegraphics{ours_sup/animal_pose_fits/segs/2007_008222.jpg} & 
        %\hspace{\spacercomp} 
        \includegraphics{ours_sup/animal_pose_fits/orig/2007_009605.jpg} &
        \includegraphics{ours_sup/animal_pose_fits/fit/2007_009605.jpg} &
        \includegraphics{ours_sup/animal_pose_fits/model/2007_009605_crop.jpg} &
        \includegraphics{ours_sup/animal_pose_fits/joints/2007_009605.jpg} &
        \includegraphics{ours_sup/animal_pose_fits/segs/2007_009605.jpg} \\ 
        
        & (a) & (b) & (c) & (d) & (e) & 
        %\hspace{\spacercomp} 
        (a) & (b) & (c) & (d) & (e) \\

    \end{tabularx}
    %
    \caption{%
    \textbf{Qualitative results on StanfordExtra and Animal Pose~\cite{animalpose}.} 
        For each sample we show: (a) input image, (b) predicted 3D mesh, 
        (c) canonical view 3D mesh, (d) reprojected model joints and 
        (e) silhouette reprojection error.
    }
    \label{fig:qualresults_sup}
\end{figure*}
