%!TEX root = ../thesis.tex
%*******************************************************************************
%****************************** Second Chapter *********************************
%*******************************************************************************

\chapter{Related Work}

\ifpdf
    \graphicspath{{Chapter3/Figs/Raster/}{Chapter3/Figs/PDF/}{Chapter3/Figs/}}
\else
    \graphicspath{{Chapter3/Figs/Vector/}{Chapter3/Figs/}}
\fi

\section{Introduction}

This section will discuss related work

% \section{Obtaining animal test data}
%     A significant drawback due to the lack of available training data is that state-of-the-art segmentation pipelines require a wealth of (RGB Input, Segmentation) and (IR Input, Segmentation training pairs which are not readily available for animal targets. To resolve this, a collaborative project is underway between GSK and Texuna to design a bespoke camera module which can be fit into rodent, dog, mini-pig and rabbit enclosures. Figure \ref{fig:texuna_cage} shows a recent prototype design. The cameras are able to flip between RGB and IR capture modes and use IP camera technology, which is supported by a number of off-the-shelf recording systems. Once animal data is successfully recorded, it will be passed to human annotators to create a suitable dataset to train a segmentation network. Until this has been obtained, the system is designed on the premise of receiving perfect silhouette segmentations. 

%     % Talk a little more about the data -- problems (all outside?), occlusions, pets, number of pictures per animal (distribution) 
%     For testing, there are examples of real-world animal segmentations that form a satisfactory set for system testing. Between the Weizmann Horse Dataset~\cite{weizmann}, the IIIT-OXFORD PET dataset~\cite{oxfordpetdata} and animal superclasses from popular datasets such as MS Coco~\cite{lin2014microsoft} and PASCAL VOC~\cite{everingham2010pascal}, there are approximately one thousand RGB-segmentation pairs for independent images (i.e.\ not part of an available video sequence). It is worth noting that the distribution of images is heavily weighted towards cats, dogs and horses over other animal species. Moreover, images tend to be taken in outdoor environments with side-on animal views. This bias has been partially resolved by sourcing an additional 1500 frames from YouTube and the BBC Blue Planet II documentary series\footnote{Appropriate permissions have been sought where applicable}. These sequences were all segmented by hand, using the RotoBrush tool provided by the Adobe After-Effects~\cite{Bai:2009:VSC} package.  


\section{Parameterizing animals}
    Given that human tracking is now an established computer vision subfield, and the growing interest in analysing human behaviour from CCTV camera tracking systems, it is natural to ask whether the techniques used in this work transfer to the animal case. As an emerging research field, animal tracking presents many challenges in common with human gait and pose tracking problems, particularly in accurately monitoring morphable objects which frequently self-occlude. However, notable additional challenges are posed by the large shape and texture variation between animal tracking candidates and also due to the lack of available training data which could otherwise be employed to train deep neural networks. An advantageous aspect of tracking animals over humans is the simple fact that animals tend not to wear clothing, which in humans causes significant shape and appearance variability.

    The previous chapter discussed the primary objective of this work, which is to recover full 3D shape and pose from a live input video sequence exhibiting an animal subject. As explained, the major challenge common to all methods operating on monocular RGB input is to resolve the inherent depth ambiguity associated with recovering a 3D model from 2D input. Competitive methods achieve this by relying on strong motion cues~\cite{kinect_fusion} or (if available) by incorporating strong prior knowledge of the tracking target. Strong shape and pose priors (e.g. body part configuration, acceptable body part lengths, likely joint positions etc.) are available for this problem, so this report will focus on analysing these methods in the literature.
    
    All solutions face an important design decision, which is to make a distinction between features of an input sequence the system should aim to model and to which it should remain invariant. For example, nearly all human systems aim to model the angle between a tracking target's upper and lower leg region, but nearly all will attempt to remain invariant to skin colour variation between candidates. The next two sections discuss examples of systems in which this decision has been made differently, generally according to the intended real-world application.
 
\section{Skeletal fitting}
    The first class of techniques, known as \textit{skeletal fitting} methods, recover detailed pose attributes from the target but learn only weak shape attributes. As these methods only ever return a skeletal outline, apart from basic limb measurements, no other shape detail (e.g. surface definition, object density etc.) is obtained. However, it is important to note that this is often perfectly satisfactory, and wholly dependent on the intended use-case. In particular, such techniques have found numerous applications in controllerless gaming and for other human-computer interaction purposes. Early approaches worked by building statistical models of limb lengths and poses using freely available motion capture data~\cite{barron2001estimating}. These could then be used to adapt a digital skeleton to fit each frame of an input video sequence. 
    
    \subsection{Kinect body joint prediction}
    Shotton et al. \cite{kinectpaper} extended this approach by designing a human skeletal tracking capability which was later incorporated into the SDK used by the Microsoft Kinect Sensor (see Figure \ref{fig:kinect_skeleton}. A large motion capture database containing approximately 500K frames was captured from human subjects performing a wide variety of activities (e.g.\ driving, dancing, kicking, running, etc.). This dataset was then used to drive a generative body model (constituting strong prior knowledge for this problem) which could be sampled from to create synthetic depth images with dense body part labels. A random forest classifier is then used to predict these body labels on unseen examples. A per-pixel density estimator for each body part is calculated for each 3D world space coordinate based on: (1) the inferred body part probability for the projected pixel, (2) the world surface area of the pixel. Density estimators for each body part are then used in combination to localize particular body joints, which are annotated with a calculated confidence value. 

    \begin{figure}[H] % Example image
        \center{\includegraphics[width=0.25\linewidth]{human_tracking}}
        \caption{Kinect generating per-pixel joint proposals.}
        \label{fig:kinect_skeleton}
    \end{figure}

    \subsection{State-of-the-art approaches}
    Most modern approaches employ a two-stage method; the first to predict 2D joint positions on an input image, followed by a step that `lifts' these to a 3D pose. Note that both steps are ambiguous. Determining joint positions is a task made challenging due to large variations in visual appearence, commonly due to clothing, body shape and camera view. As explained by Toshev and Szegedy~\cite{toshev2014deeppose}, even with perfect joint locations the subsequent lifting step is also ill-posed, as the space of consistent 3D poses for given 2D landmark locations is infinite. This is typically resolved using strong prior knowledge which usually takes the form of 3D geometric pose priors and temporal or structural constraints. Examples of such systems include DeepPose~\cite{toshev2014deeppose}, an approach which employs a CNN to reason jointly about 2D landmark detection and 3D pose estimation from single RGB images. Pishchulin et al.~\cite{pishchulin2016deepcut} later introduced DeepCut which extends DeepPose to the multi-person case. Both systems are trained on large body joint databases. 

    However, some direct techniques exist which do not require an initial 2D joint prediction. These include methods that directly regress to a 3D pose~\cite{tekin2016direct}. However, these typically rely on an annotated set of 3D joint labels, which can be difficult and costly to obtain, or being able to build a representative synthetic dataset, which is non-trivial task.

\clearpage
\section{Model fitting}
    As previously mentioned, despite the suitability for a range of applications, skeletal fitting methods perform relatively weak shape modelling and therefore provide insufficient detail for animal diagnostic purposes. This section discusses model fitting approaches which are more applicable to the set objectives.

    \subsection{Form of the template prior}
    A polygon mesh $M = (V, T)$ is a collection of vertices, edges bound by vertex pairs, and polygons bound by sequences of edges and vertices~\cite{smith2006vertex}. Although other convex shapes are allowed, mesh polygons will always be considered triangular (and hence referred to as \emph{triangles}) unless explicitly stated otherwise. An example mesh is shown in Figure~\ref{fig:polygon_mesh}. 
    
    \begin{figure}[H] % Example image
        \center{\includegraphics[width=0.5\linewidth]{dolphin_mesh}}
        \caption{A polygon mesh~\cite{polygon_mesh}.}
        \label{fig:polygon_mesh}
    \end{figure}

    \subsection{Formalizing model fitting}
    Model (or mesh) fitting encompasses a set of methods that work by adapting a 3D template mesh to either an input image or input video sequence. Such techniques therefore return a full 3D model intended to faithfully reconstruct the performance given by the tracking target, although the accuracy of this reconstruction is heavily conditioned on the quality of the method.

    \subsubsection{Dense correspondence}
    Model fitting algorithms are typically provided with point correspondences between the template mesh and input images in order to help constrain the optimization. Correspondences are either provided by a human annotator or predicted by a discriminative machine learning model. A particular area of interest for this project is to evaluate whether `dense' correspondences, in which \emph{every} input image pixel is asssigned a matching mesh point, improves model fitting. On one hand, accurately-predicted per-pixel correspondences further constrains an optimizer and should therefore lead to satisfactory convergence in fewer iterations. However, systematic or large correspondence errors may lead the optimizer to falling into unsatisfactory local minima.
    
    \subsection{Mesh deformation}
    The process of adapting a 3D mesh is known as \textit{mesh deformation} and is common across many computer graphics applications, particuarly those in which models are designed to represent dynamic objects. To constrain an optimization function (or simplify the animation process), it is useful to introduce priors that prevent unnatural mesh movement. Two methods for achieving this are discussed:

    \subsubsection{As Rigid as Possible}
        As Rigid as Possible (ARAP) surface deformation~\cite{sorkine2007rigid} is a distance function that measures similarity between two meshes with corresponding vertices. For two vertex sets~$V_{1}$ and~$V_{2}$, ARAP minimizes over~$N = |V|$ rotation matrices. Note~$j \sim i$ indicates vertex indices~$j$ adjacent to vertex index~$i$:

        \begin{equation}
            D(V_{1}, V_{2}) = \min_{R_{1..N}}\sum_{i=1}^{N}\sum_{j \sim i}|| (V_{1i} - V_{1j}) - R_{i}(V_{2i} - V_{2j}) ||^{2}
        \end{equation}

        This distance function can be incorporated into an energy-based optimizer as a regularization function. By considering how small vertex regions overlap, the function can be used to  discourage `unnatural movement', e.g.\ shearing effects, over mesh faces. ARAP regularizers are particularly useful in cases in which there is no prior knowledge of the mesh. Figure~\ref{fig:arap_dino} shows an example of a dinosaur mesh undergoing ARAP deformation, obtained by translating the highlighted yellow vertex.

        \begin{figure}[H] % Example image
            \center{\includegraphics[width=0.35\linewidth]{dino_arap}}
            \caption{Dinosaur mesh undergoing ARAP deformation, obtained by translating the highlighted yellow vertex. Reprinted from~\cite{sorkine2007rigid}.}
            \label{fig:arap_dino}
        \end{figure}

        \subsubsection{Skeletal Rigging and Linear Blend Skinning}
        In cases that the mesh shape is known in advance, it is common to follow a process known as \textit{rigging}, in which the mesh is augmented with a hierarchical bone structure. The point at which two bones meet is called a \emph{joint}, and these can be used to define acceptable centres of rotation for mesh deformation. It is possible to describe a distribution of joint configurations, which could be used to constrain the mesh to (in the case of human / animal subjects) anatomically achievable poses. It is also simple to define conceptual `body parts' from a rigged mesh, by considering regions between pairs of joints; for example a lower leg region can be defined between a knee and ankle joint. A simple example of a rigged 2D mesh with joints indicated by green diamonds is shown in Figure~\ref{fig:finger_model}. Note how the mesh surface deforms naturally as the joints are displaced.
        
        \begin{figure}[H]
            \centering
            \begin{subfigure}{0.48\linewidth}
            \centering
                \includegraphics[width=1\linewidth]{finger/finger1}
                \caption{Default joint positions.}
            \end{subfigure}
            \begin{subfigure}{0.48\linewidth}
            \centering
                \includegraphics[width=1\linewidth]{finger/finger2}
                \caption{Right-most joint displaced.}
            \end{subfigure}
            \begin{subfigure}{0.48\linewidth}
            \centering
                \includegraphics[width=1\linewidth]{finger/finger3}
                \caption{Central joint displaced and right-most joint displaced and rotated.}
            \end{subfigure}%
            \caption{Web application demonstrating LBS on a 2D finger mesh. Joints are denoted as green diamonds.}
            \label{fig:finger_model}
        \end{figure}

        \clearpage
        Formally, a skinned mesh consists of a set of rigged vertices $V \subseteq \mathbb{R}^3 \times \mathbb{R}^{|J|}$, a set of faces $F \subseteq V^3$ and joints $J \subseteq R^{3\times3}$. Each vertex $v = (x, s) \in V$ consists of positional coordinate $x \in \mathbb{R}^{3}$ and a weight vector $s \in \mathbb{R}^{|J|}$ which describes the level of influence each joint $j \in J$ has over its movement. Many approaches exist for assigning weights, but perhaps the simplest is to build a vector with entries corresponding to the distance from the vertex to each joint centre. Skinning weight vectors are normalized such that their entries sum to one, and for computational reasons, the number of non-zero elements is typically limited to 2 or 4. The weakness of such models is that artifacts and other unrealistic deformations can occur around the model joints, particularly for meshes that model non-linear structures such as humans. However, the technique is frequently used in computer graphics and game design when a character's shape is known ahead of time.

        To assist in explanation, Figure \ref{fig:rigged_cylinder} shows skinning weight influences from three joints within a rigged cylinder mesh. Here, $|J| = 3$ and each vertex $v_{i} = (x_{i}, s_{i}) \in V$ has a skinning weight vector $s_{i} \in \mathbb{R}^{3}$. Each model joint is assigned a distinct RGB value, shown separately in (a), (b) and (c), and together in (d) by linearly combining the colours. This linear blend colorization scheme will be frequently used in later sections of this report.

        \begin{figure}[H]
            \centering
            \begin{subfigure}{0.25\linewidth}
            \centering
                \includegraphics[width=1\linewidth]{wonky_pole/lower_bone}
                \caption{Lower joint.}
            \end{subfigure}%
            \begin{subfigure}{0.25\linewidth}
            \centering
                \includegraphics[width=1\linewidth]{wonky_pole/middle_bone}
                \caption{Middle joint.}
            \end{subfigure}%
            \begin{subfigure}{0.25\linewidth}
            \centering
                \includegraphics[width=1\linewidth]{wonky_pole/upper_bone}
                \caption{Upper joint.}
            \end{subfigure}%
            \begin{subfigure}{0.25\linewidth}
            \centering
                \includegraphics[width=1\linewidth]{wonky_pole/linear_blend}
                \caption{Linear blend.}
            \end{subfigure}%
            \caption{A rigged cylinder with $|J| = 3$ and where each vertex $v_{i} = (x_{i}, s_{i}) \in V$ has a skinning weight vector $s_{i} \in \mathbb{R}^{3}$.}
            \label{fig:rigged_cylinder}
        \end{figure}

        Figure \ref{fig:rigged_quadruped} shows a more complex rigged quadruped mesh with $|J| = 25$ with skinning weight influences again shown by the linear blend colorization scheme. Again, each joint is assigned a unique RGB value and a vertex's colour is calculated by linearly combining joint colours with skinning weight vectors given by the $\{s_{i}\}$. A triangle's colour is then generated by averaging the colours given for the three surrounding vertices.

        \begin{figure}[H] % Example image
            \center{\includegraphics[width=0.5\linewidth]{linear_blend_bold_bones}}
            \caption{A rigged quadruped with $|J| = 25$ and where each vertex $v_{i} = (x_{i}, s_{i}) \in V$ has a skinning weight vector $s_{i} \in \mathbb{R}^{3}$. Visualization uses the linear blend colorization scheme in which each joint is assigned a unique RGB value.}
            \label{fig:rigged_quadruped}
        \end{figure}

        Once a mesh has been suitably rigged, there are a number of options (e.g. Linear Blend Skinning (LBS), Dual Quaternions~\cite{kavan2007skinning} etc.) for applying a particular mesh deformation. Typically, a user assigns a transformation (in this case comprising a rotation and transformation) to each `joint' and the updated positions $\bar{x_{i}}$ of the remaining vertices $v_{i}$ with original positions $x_{i}$ are then calculated. The original transformation for each joint (i.e. before the deformation) is expressed as a matrix $U_{j}$. The transformation after the deformation has been applied is captured by $D_{j}$. Note that $s_{ij}$ denotes the skinning weight influence of joint~$j \in J$ on vertex $v_{i} \in V$.
        
        The updated positions $\bar{x_{i}}$ can then be calculated by LBS:

        \begin{equation}
            \bar{x_{i}} = \sum_{j=1}^{|J|}s_{ij}D_{j}U_{j}^{-1}x_{i}
        \end{equation}

        \clearpage
        \subsubsection{Rendering}
        The process of generating a 2D image from a 3D polygon mesh is known as rendering and can be achieved through a process known as raytracing. Raytracing is a rendering technique able to generate photorealistic 2D images from the scene. It can be considered the opposite process by which the human eye perceives the world, as this method involves lines being cast outwards, beginning at a point known as the \emph{camera origin}. Figure \ref{fig:raycasting} shows a typical set up, in which rays are cast from the camera origin through each pixel on the image plane. The colour for the pixel is obtained by following the ray through the scene until a light source or non-reflective surface is reached, taking into account any reflections or non-opaque scene items. Due to the considerable comptuation required, the operation is often parallelized and assigned to the GPU. However, the technique is typically considered unsuitable for real-time rendering of complex scenes (due to complex ray paths) or when high resolution images (many rays required) are needed. However, for this work, scenes are typically made up of a single non-reflective, solid mesh surface and contain no complex elements (e.g.\ shadows, non-constant lighting.

        \begin{figure}[H] % Example image
            \center{\includegraphics[width=0.5\linewidth]{ray_trace}}
            \caption{Diagram showing raycast rendering.~\cite{rendering}.}
            \label{fig:raycasting}
        \end{figure}

        It is also worth noting that the standard method for raycasting is not differentiable, causing problems for differentiable optimizers (including neural networks). However, alternative rendering methods~\cite{loper2014opendr} are available for these purposes.

        \clearpage
        \section{Methods for model fitting}
        The following section describes an number of existing model fitting methods.

        \subsection{Fitting a rigged template mesh to rigged dense correspondences}
        Taylor et al. demonstrate a model fitting approach that operates on a rigged 3D human mesh~\cite{taylor2012vitruvian}. Their aim is to learn a set of pose parameters $\theta \in \mathbb{R}^{d}$ so as to explain a set of image points $D = \{x_{i}\}_{i=0}^{n}$. Data points $x_{i} \in \mathbb{R}^{3}$ are collected from a calibrated depth camera. Once these pose parameters are learnt, the mesh is deformed according to the LBS algorithm defined above.

        The template mesh contains $|J| = 13$ joints, and $m$ skinned vertices ${V} = \{v_{i}\}_{i=1}^{m}$. Again, each vertex $v_{i} \in V$ is defined as:
        
        \begin{equation}
            v_{i} = (x_{i}, s_{i})
        \end{equation}
        where $x_{i} \in \mathbb{R}^{3}$ represents the base 3D vertex positions in a canonical pose $\theta_{0}$ and the $s_{i} \in \mathbb{R}^{|J|}$ are skinning weight vectors. It is possible to define a mesh induced by a pose $S(\theta) = (V, T)$ for vertices $V$ and triangles $T$. Due to the resemblance of the mesh surface induced by the canonical mesh pose $\theta_{0}$ and Da Vinci's Vitruvian man~\cite{davinci}, this surface is referred to as the \emph{Vitruvian Manifold}, and is shown in Figure~\ref{fig:vitruvian_man}. 
        
        \begin{figure}[H] % Example image
            \center{\includegraphics[width=0.95\linewidth]{vitruvian_man}}
            \caption{(a) Vitruvian Man by Leonardo da Vinci~\cite{davinci} and (b) the Vitruvian Manifold reprinted from~\cite{taylor2012vitruvian}.}
            \label{fig:vitruvian_man}
        \end{figure}

        The primary contribution of this paper is the design of a model able to predict \emph{dense correspondences} between the 3D canonical mesh and input 3D images. In other words, \emph{every} body pixel on an input image is regressed to a point on the vitruvian manifold mesh. The authors demonstrate the accuracy of these correspondences is sufficient for \emph{one-shot learning}, meaning there is no need to recalculate correspondences after a subsequent optimization step. The reason for this is the strength of the core error term which penalizes the sum of errors between image points $\{x_{i}\}_{i=0}^{n}$ and determined mesh correspondences $U = \{u_{i}\}_{i=0}^{n} \subseteq V$:

        \begin{equation}
            E_{\text{data}}(\theta,U) =\sum_{i=1}^{n}s_{i} \cdot d(x_{i}, M(u_{i}; \theta))
        \end{equation}
        where $M(u_{i}, \theta)$ is the position of vertex $u_{i}$ on the vitruvian manifold mesh after having been displaced by an LBS deformation with respect to the pose~$\theta$. 
        
        The sheer quantity of correspondences greatly constrain their optimizer which works well, even on challenging input images. Much of this report focuses on how this paper can be extended to work for animal subjects, incorporating deep learning correspondence prediction and working from monocular RGB input data.

        \subsection{Non-skeletal fitting to animal video sequences}
        Stebbing et al.~\cite{arap_stebbing} introduce a technique capable of fitting a template mesh to live video sequences for a range of different animal species. Some user interaction is required in order to segment the animal from the background and to provide sparse 3D-mesh-to-2D-image key point correspondences. This work only operates on input video sequences (rather than single frames), so a number of temporal terms are incorporated that encourage sensible inter-frame model deviation. The system requires an annotated input template mesh representative of the target animal species. Note that this work does not require the template mesh to have an inner skeletal structure. However, the user assists an ARAP-style term by assigning each mesh vertex $v_i$ to one of $M$ groups which share a set of basis rotations $B_{m}$. 

        \begin{figure}[H]
            \centering
            \begin{subfigure}{0.5\textwidth}
            \centering
                \includegraphics[height=0.5\linewidth]{arapsfm/arap_annotated_template}
                \caption{Template mesh with joint movement constraints.}
            \end{subfigure}%
            \begin{subfigure}{0.5\textwidth}
            \centering
                \includegraphics[height=0.5\linewidth]{arapsfm/arap_point_tracks}
                \caption{Example of user supplied point tracks.}
            \end{subfigure}%
            \caption{User input required for the deformable mesh animation algorithm, reprinted from~\cite{arap_stebbing}.}
            \label{fig:arap_user}
        \end{figure} 

        \clearpage
        Through reasonably accurate pose fitting and by allowing some pose-invariant shape deformation, this work produces smooth meshes which are often a good match to the input video. Moreover, their experimentation shows that ARAP is a useful prior for reconstructing articulated, non-rigid motion in instances that an internal skeleton is a priori unknown. However, the shape attributes for the reconstructed model are not particularly accurate, which results in frequent errors appearing at internal occluding contours. In addition, the large non-convex optimization algorithm is an expensive operation, taking around 1 minute per video frame on a standard Linux workstation.
        
        Results showing this work fitting a crude dog template mesh to a sample video obtained from YouTube are shown previously in Figure \ref{fig:intro_arap_output}. Figure \ref{fig:arap_output} shows another example, which operates on a template impala mesh.

        \begin{figure}[H]
            \center{\includegraphics[width=0.95\linewidth]{arapsfm/arap_impala}}
            \caption{Example of an impala template being fit to input video sequence, reprinted from~\cite{arap_stebbing}}
            \label{fig:arap_output}
        \end{figure}

    \subsection{Learning animal shape from unrelated 2D images}
        Cashman and Fitzgibbon~\cite{cashman2013shape} introduce an optimization technique able to recover a parameterized, morphable 3D model from unrelated 2D images depicting examples of the target class. The method requires user-supplied 2D object outlines and point constraints for each image, and a single rigid mesh for the entire object class. The authors demonstrate recovering an 8-parameter morphable dolphin model from 32 images obtained from Google. To reduce required user activity, it is reasonable to assume that given sufficient labelled training data, it would be simple to manipulate a convolutional network architecture able to perform foreground / background segmentation and identify human key points (say, joints) for the desired object class. The system achieves impressive results when optimizing over both pose and shape parameters across a range of object classes, but struggles for articulated models such as polar bears.

        \begin{figure}[H] % Example image
            \center{\includegraphics[width=0.7\linewidth]{dolphins}}
            \caption{8-parameter dolphin model with annotated contour (left) and contour generators (middle and right).}
            \label{fig:cashman_fitzgibbon}
        \end{figure}

    \subsection{Fitting to an articulated hand model}
        Given the availability of strong shape and pose priors, articulated hand tracking aptly demonstrates the advantage of model fitting approaches. Again, it is first necessary to decide how the human hand should be parameterized, i.e. what an optimizer should specifically aim to learn. Similar to the case with the full human body, the aim is again to adapt a mesh (although this time of a hand) to reproduce a performance given by a real human hand either in still frames or from an input video sequence. Many modern approaches follow a hand parameterization given by Khamis et al.~\cite{Khamis_2015_CVPR} using a pose vector $\theta \in \mathbb{R}^{28}$ that includes global translation and rotation, one adbuction and three flexion variables for each finger digit, and one abduction and flexion parameter for the wrist and forearm. An example hand tracking result can be seen in Figure~\ref{fig:hand_tracking}. 

        \begin{figure}[H] % Example image
            \center{\includegraphics[width=0.85\linewidth]{hand_tracking}}
            \caption{Example of articulated hand tracking, reprinted from~\cite{taylor2016efficient}.}
            \label{fig:hand_tracking}
        \end{figure}

    \subsection{Data-driven body models}
        Data-driven statistical body models built from large database of human 3D scans are receiving increasing attention from the research community. Having been trained on examples of real humans of a range of different shapes and adopting various poses, these models capture subtle details that is hard to encode explicitly. In part due to a good choice of training candidates, SCAPE~\cite{anguelov05scape}, FAUST~\cite{bogo2014faust} and SMPL~\cite{loper15smpl} models exemplify this technique and are able to account for many body shapes, poses and non-rigid deformations such as muscle bulging due to joint articulation. The quality of such models is such that a user can construct visually-realistic bodies that were never present in the original data. A noteable drawback of such approaches is the required time and financial investment in conducting the data capture and the subsequent need to align each scan.

        SMPL first learns how human beings deform through pose changes using 1786 high-resolution 3D scans of different subjects in a wide variety of poses. Following alignment to a template mesh, a linear model for each biological gender is created from the CAESAR dataset \cite{robinette2002civilian} using principal component analysis (PCA). SMPL was motivated by the ambition to generate a realistic data-driven human body model which can be rendered in real-time using standard engines, such as Unity~\cite{unity2017} or Blender~\cite{blender2017}. Having been designed for animation, SMPLs base template has a number of useful qualities for this work; the underlying mesh is a clean structure and comprises relatively few polygons. A novelty of this model is that it encodes explicit and meaningful body joint positions. Some sample SMPL meshes are shown in Figure \ref{fig:smpl_model}.

        \begin{figure}[H] % Example image
            \center{\includegraphics[width=0.5\linewidth]{smpl_wbg}}
            \caption{SMPL model showing pose-invariant shape changes, reprinted from~\cite{loper15smpl}.}
            \label{fig:smpl_model}
        \end{figure}

        A similar technique to that used to build the SMPL model has been recently used to build a Skinned Multi-Animal Linear Model (SMAL)~\cite{zuffi2017menagerie}, a generative animal model exhibiting realistic 3D shape (see Figure \ref{fig:smal_model_shape}) and pose (see Figure \ref{fig:smal_model_poses}). Due to the lack of available motion capture data for animal subjects, the SMAL model is learnt from a set of $41$ 3D scans of toy figurines in arbitrary poses. The figurines span five quadruped families, and included examples of lions, cats, tigers, dogs, horses, any many more, although notably for this work no rodent toys were included. The paper introduces a new technique to accurately align each toy scan to a common template, allowing the shape space to be learnt.

        \begin{figure}[H]
            \centering
            \begin{subfigure}{0.3\linewidth}
            \centering
                \includegraphics[width=1\linewidth]{smal/default}
                \caption{Default SMAL mesh.}
            \end{subfigure}%
            \begin{subfigure}{0.3\linewidth}
            \centering
                \includegraphics[width=1\linewidth]{smal/horse}
                \caption{SMAL in horse shape.}
            \end{subfigure}%
            \begin{subfigure}{0.3\linewidth}
                \centering
                    \includegraphics[width=1\linewidth]{smal/lion}
                    \caption{SMAL in lion shape.}
            \end{subfigure}%
            \caption{SMAL with varying shape parameters.}
            \label{fig:smal_model_shape}
            \end{figure}
    
            \begin{figure}[H]
            \centering
            \begin{subfigure}{0.3\linewidth}
            \centering
                \includegraphics[width=1\linewidth]{smal/pose_1}
            \end{subfigure}%
            \begin{subfigure}{0.3\linewidth}
            \centering
                \includegraphics[width=1\linewidth]{smal/pose_2}
            \end{subfigure}%
            \begin{subfigure}{0.3\linewidth}
                \centering
                    \includegraphics[width=1\linewidth]{smal/pose_3}
            \end{subfigure}%
            \caption{SMAL with varying pose parameters.}
            \label{fig:smal_model_poses}
        \end{figure}

        From the paper, SMAL is defined as a function $M(\beta, \theta, \gamma)$ parameterized by pose-invariant shape $\beta$, pose $\theta$ (including global rotation) and global translation $\gamma$. The function returns a triangulated surface comprising $6890$ vertices. SMAL contains $41$ shape parameters $\beta$ which are coefficients of a low-dimensional shape space. There are three pose parameters for each of the $32$ body joints and an additional three to express the global rotation. Global translation $\gamma$ is expressed by a further three parameters.

        \subsubsection{Fitting the SMPL mesh to human images}
        SMPLify \cite{bogo16keep} is a fully-automated optimization approach that uses predicted human joint positions to constrain a optimizer that fits the aforementioned SMPL model to RGB input images. It first makes use of the DeepCut CNN to predict 2D human body joints $J_{\text{est}}$ on input frames. For each 2D joint $i$ the CNN is able to provide a confidence value $w_i$ for the joint's position. The optimization begins by first solving for global rotation (i.e. $\theta_{0..2}$) and global translation $\gamma$ by fitting a small number of 2D torso points $J_{\text{torso}} \subset J_{\text{est}}$ to the data. The user is expected to provide a value for the focal length $f$. Then, the full optimization takes place, fitting 3D pose and shape to all 2D joints by minimizing the following objective function which comprises five error terms:

        \begin{equation}
        E(\beta, \theta) = E_{J}(\beta, \theta; K, J_{\text{est}}) + \lambda_{\theta}E_{\theta}(\theta) + \lambda_{\alpha}E_{\alpha}(\theta) + \lambda_{\text{sp}}E_{\text{sp}}(\theta; \beta) + \lambda_{\beta}E_{\beta}(\beta)
        \end{equation}
        where $\lambda$ terms are the scalar weights. The term $E_{J}$ is often referred to as the \textit{data} term, as it places most emphasis on constraining the model to the input sensory data. The job of this term is to penalize the weighted 2D distance between estimated joints $J_{\text{est}}$ and corresponding projected SMPL joints. In practice, this projection takes place using the OpenDR differentiable rendering framework to ensure the final formulation remains differentiable:

        \begin{equation}
            E_{J}(\beta, \theta; K, J_{\text{est}}) = \sum_{joint j} w_{j} \rho(\Pi_{K}(R_{\theta}(J(\beta)_j)) - J_{\text{est}, j})
        \end{equation}
        where $J(\beta)$ is a function which predicts 3D body joints from body shape and $R_{\theta}(J(\beta))$ therefore denote posed 3D joints.

        The remaining terms are now briefly discussed:
        \clearpage
        \begin{itemize}
            \item $E_{\theta}(\theta)$ is referred to as a \textit{pose prior} which favours more likely poses by assigning large punishment to those that deviate from known poses collected from a large dataset.
            \item $E_{\beta}(\beta)$ is referred to as a \textit{shape prior} which favours more likely pose-invariant shape configurations by assigning large punishment to those that deviate from known shapes collected from a large dataset. 
            \item $E_{\alpha}(\theta)$ is a \textit{joint limit} prior which ensures particular joints remain within acceptable angle limits. For example, a knee joint in a human model should be prohibited from bending more than 5 degrees upwards.
            \item $E_{sp}(\theta; \beta)$ is an \textit{interpenetration} term, which can only be defined in such shape modelling approaches. Using both shape and pose from the model, it is possible to determine if any limbs are self-intersecting, or intersect other parts of the body and assign appropriate penalty.
        \end{itemize}

        An example result can be seen in Figure \ref{fig:smplify}:

        \begin{figure}[H] % Example image
            \center{\includegraphics[width=0.95\linewidth]{fitting_smpl}}
            \caption{SMPLify: Fitting the SMPL model to the Leeds Sports Dataset.}
            \label{fig:smplify}
        \end{figure}

        \subsubsection{Fitting the SMAL mesh to animal images}
        The SMAL paper briefly discusses a modification to the SMPLify approach in order to fit the SMAL model to RGB animal input images. The terms are largely the same, although the interpenetration term is omitted and joint positions are provided manually, rather than being predicted by a CNN. Finally, the optimizer requires a pre-segmented (i.e.\ silhouette) image which is also supplied by a user. An approach discussed in Chapter 4 builds on this work, so an in-depth description of this method is ommited here. However, an example result showing the result of the optimizer fitting the SMAL mesh to an RGB image of a fox can be seen in Figure \ref{fig:smalify}. Note that the whole optimization process takes around 1 minute per frame.

        \begin{figure}[H] % Example image
            \center{\includegraphics[width=0.95\linewidth]{fitting_smal}}
            \caption{Fitting SMAL to a hand segmented animal, reprinted from~\cite{zuffi2017menagerie}.}
            \label{fig:smalify}
        \end{figure}

    \subsection{Direct regression}
    Very recently, deep learning techniques have been employed to solve the entire optimization problem by directly regressing to shape and pose parameters. Tekin et al.~\cite{tekin2016direct} introduce a convolutional network trained on the Human3.6m dataset~\cite{lin2014microsoft} that directly regresses to a human pose defined in terms 3D locations $y \in \mathbb{R}^{3J}$ of $J$ body joints relative to a root joint. Tan et al.~\cite{tan17indirect} introduce an approach that directly regresses to SMPL parameters from synthetic images, ensuring suitable image jitters are applied to promote generality to real-world images. The method is termed Indirect Learning, and is trained from real human images with no known corresponding SMPL parameters. An autoencoder network is introduced, and a decoder first trained from synthetic (SMPL parameter, rendered image) pairs to construct an automatic renderer. This part of the network is then frozen, before the entire autoencoder is trained on many segmented human images that optimize the encoder to real-world examples. The end result is a process that is able to predict SMPL parameters from real-world human images. An example result is shown in Figure~\ref{fig:indirect_learning}.

    \begin{figure}[H] % Example image
        \center{\includegraphics[width=0.75\linewidth]{indirect_learning}}
        \caption{Indirect learning method regressing to SMPL parameters from an RGB video sequence. Reprinted from~\cite{tan17indirect}.}
        \label{fig:indirect_learning}
    \end{figure}


