% \section{Related work}\label{s:related}

% There is ample literature on recovering the pose of 3D models from images.
% We break this into five categories: methods that reconstruct 3D points directly, methods that reconstruct the parameters of a 3D model of the object via optimization, methods that do the latter via learning-based regression, hybrid methods and methods which deal with uncertainty in 3D human reconstruction.

% \subsection{Reconstructing 3D body points without a model.}

% Several papers have focused on the problem of estimating 3D body points from 2D observations~\cite{anguelov05scape,mehta17vnect,rogez18lcr-net,sun18integral,kolotouros19convolutional}.
% Of these, {}\citet{martinez17a-simple} introduced a particularly simple pipeline based on a shallow neural network.
% In this work, we aim at recovering the full 3D surface of a human body, rather than only lifting sparse keypoints.

% \subsection{Fitting 3D models via direct optimization.}

% Several methods \emph{fit} the parameters of a 3D model such as SMPL~\cite{loper15smpl} or SCAPE~\cite{anguelov05scape} to 2D observations using an optimization algorithm to iteratively improve the fitting quality.
% While early approaches such as~\cite{guan09estimating,sigal08combined} required some manual intervention, the SMPLify method of~\citet{bogo16keep} was perhaps the first to fit SMPL to 2D keypoints fully automatically.
% SMPL was then extended to use silhouette, multiple views, and multiple people in~\cite{lassner17unite,huang17towards,zanfir18monocular}.
% Recent optimization methods such as~\cite{joo18total,pavlakos19expressive,xiang19monocular} have significantly increased the scale of the models and data that can be handled.

% \subsection{Fitting 3D models via learning-based regression.}

% More recently, methods have focused on regressing the parameters of the 3D models directly, \emph{in a feed-forward manner}, generally by learning a deep neural network~\cite{tan17indirect,tung17self-supervised,omran18neural,pavlakos18learning,kanazawa18end-to-end}.
% Due to the scarcity of 3D ground truth data for humans in the wild, most of these methods train a deep regressor using a mix of datasets with 3D and 2D annotations in form of 3D MoCap markers, 2D keypoints and silhouettes. Among those, HMR of~\citet{kanazawa18end-to-end} and GraphCMR of~\citet{kolotouros19convolutional} stand out as particularly effective.

% \subsection{Hybrid methods.}

% Other authors have also combined optimization and learning-based regression methods.
% In most cases, the integration is done by using a deep regressor to initialize the optimization algorithm~\cite{sigal08combined,lassner17unite,rogez18lcr-net,pavlakos18learning,varol18bodynet}.
% However, recently~\citet{kolotouros19learning} has shown strong results by integrating the optimization loop in learning the deep neural network that performs the regression, thereby exploiting the weak cues available in 2D keypoints.

% Use this section as an excuse to explain VAEs and MDNs
\subsection{Modelling ambiguities in 3D human reconstruction.}

Several previous papers have looked at the problem of modelling ambiguous 3D human pose reconstructions. Early work includes~\citet{kinematic-jump-processes}, ~\citet{tracking-3d-human-figures} and \citet{density-prop}. 

% To be fair, we don't learn a prior directly on the SMPL parameters.

% They model distributions conditioned on the parent of each joint in the kinematic tree and show that this prior can be used to obtain improved parametric fits.
More recently,~\citet{akhter15pose-conditioned} learn a prior over human skeleton joint angles (but not directly a prior on the SMPL parameters) from a MoCap dataset.~\citet{li19generating} use the Mixture Density Networks model of~\cite{bishop94mixture} to capture ambiguous 3D reconstructions of sparse human body keypoints directly in physical space.
\citet{sharma19monocular} learn a conditional variational auto-encoder to model ambiguous reconstructions as a posterior distribution; they also propose two scoring methods to extract a single 3D reconstruction from the distribution.
% Finally,~\citet{cheng19occlusion-aware} tackle the problem of video 3D reconstruction in the presence of occlusions, and show that temporal cues can be used to disambiguate the solution.
% While our method is similar in the goal of correctly handling the prediction uncertainty, we differ by applying our method to predicting \emph{full mesh} of the human body.
% This is arguably a more challenging scenario due to the increased complexity of the desired 3D shape.

% Several previous papers have looked at the problem of modelling ambiguity in 3D human pose reconstruction. Early work includes~\citet{kinematic-jump-processes}, ~\citet{tracking-3d-human-figures} and \citet{density-prop}. More recently, ~\citet{akhter15pose-conditioned} learn a prior over human skeleton joint angles from a MoCap dataset.
% %(but not directly a prior on the SMPL parameters)
% % They model distributions conditioned on the parent of each joint in the kinematic tree and show that this prior can be used to obtain improved parametric fits.
% More recently,~\citet{li19generating} use the Mixture Density Networks model of~\cite{bishop94mixture} to capture ambiguous 3D reconstructions of sparse human body keypoints directly in physical space.
% \citet{sharma19monocular} learn a conditional variational auto-encoder to model ambiguous reconstructions as a posterior distribution; they also propose two scoring methods to extract a single 3D reconstruction from the distribution.

\citet{cheng19occlusion-aware} tackle the problem of video 3D reconstruction in the presence of occlusions, and show that temporal cues can be used to disambiguate the solution.
While our method is similar in the goal of correctly handling the prediction uncertainty, we differ by applying our method to predicting \emph{full mesh} of the human body.
This is arguably a more challenging scenario due to the increased complexity of the desired 3D shape.

Finally, some recent concurrent works also consider building priors over 3D human pose using normalizing flows.~\citet{xu-2020-cvpr} release a prior for their new GHUM/GHUML model, and~\citet{weakly-supervised-normflow} build a prior on SMPL joint angles to constrain their weakly-supervised network. Our method differs as we learn our prior on 3D SMPL joints.

% \paragraph{Learning priors over 3D human pose}
% Early work 

% \begin{enumerate}
%     \item From Sminchescu, this paper looks at the fundamental ambiguities in pose
%     https://hal.inria.fr/inria-00548223/document. 
%     Kinematic skeleton of articulated joints controlled by angular joint parameters, covered by 'flesh' built from superquadric ellipsoids with additional global deformations. The superquadric surfaces are discretized into 2D meshes and the mesh nodes are mapped to 3D points. They evaluate p(x|r) where x are model parameters and r are 'nearby' image features._
%     \item And this paper from Sidenbladh represented a distribution over 3D poses and exploited this for tracking
%     http://files.is.tue.mpg.de/black/papers/eccv00.pdf
%     Body is modelled as a configuration of 9 cylinders and 3 spheres. Does this count as a mesh? Bayesian model of shape, appearance and motion (temporal). They do formulate p(pose, motion, shape | image).
%     \item In terms of discriminative (learning) methods that also represent ambiguity in poses, there is another work by Sminchisescu on "Discriminative Density Propagation for 3D Human Motion Estimation"
%     http://research.cs.rutgers.edu/~kanaujia/MyPapers/CVPR2005.pdf
% \end{enumerate}

% \begin{enumerate}
%     \item GHUM \& GHUML: Generative 3D Human Shape and Articulated Pose Models (CVPR20)
%     Hongyi Xu, Eduard Gabriel Bazavan, Andrei Zanfir, William T Freeman, Rahul Sukthankar, Cristian Sminchisescu.
%     Authors introduce a 3D kinematic pose prior for their new GHUM/GHUML generative human body model, based on normalizing flows. 
%     In this work, we learn a similar prior for SMPL although we represent pose in terms of skeleton positions rather than the SMPL angles directly.

%     \item Weakly Supervised 3D Human Pose and Shape Reconstruction with Normalizing Flows (ECCV20)
%     Andrei Zanfir, Eduard Gabriel Bazavan, Hongyi Xu, William T Freeman, Rahul Sukthankar, Cristian Sminchisescu
%     Approach learns a similar normalizing flow pose prior, based on the SMPL joint angles. We found empiracle improvements by basing our 
%     prior on 3D joint positions dervied from the SMPL pose parameters, rather than on the joint angles directly.
% \end{enumerate}

