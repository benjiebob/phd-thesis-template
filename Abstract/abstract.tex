% ************************** Thesis Abstract *****************************
% Use `abstract' as an option in the document class to print only the titlepage and the abstract.
\begin{abstract}
    TODO.
    Across many sectors concerned with animal husbandry, there is growing support for a system able to continuously monitor captive animals. Within farmyards, zoos, veterinary centres, animal research facilities and many others, humans typically take responsibility for identifying signs of disease or distress within their animal populations. While this can be effective, a significant challenge is posed when a small number of humans are expected to care for large animal groups.
  
    This report discusses the development of a system to track, monitor and react to signs of poor physiological and psychological health among captive animals. In this work, it is proposed that a useful component of such a system would be the recovery of a detailed per-frame 3D animal reconstruction from an input video sequence. This is achieved through an approach which combines discriminative machine learning with generative model fitting to recover strong shape and pose attributes. 
    
    We present a system to recover the 3D shape and motion of a wide variety of quadrupeds from video. The system comprises a machine learning front-end which predicts candidate 2D joint positions, a discrete optimization which finds kinematically plausible joint correspondences, and an energy minimization stage which fits a detailed 3D model to the image. In order to overcome the limited availability of motion capture training data from animals, and the difficulty of generating realistic synthetic training images, the system is designed to work on silhouette data. The joint candidate predictor is trained on synthetically generated silhouette images, and at test time, deep learning methods or standard video segmentation tools are used to extract silhouettes from real data. The system is tested on animal videos from several species, and shows accurate reconstructions of 3D shape and pose.
\end{abstract}
