%!TEX root = ../thesis.tex
%*******************************************************************************
%****************************** Second Chapter *********************************
%*******************************************************************************

\chapter{Conclusions}

\ifpdf
    \graphicspath{{Chapter7/Figs/Raster/}{Chapter7/Figs/PDF/}{Chapter7/Figs/}}
\else
    \graphicspath{{Chapter7/Figs/Vector/}{Chapter7/Figs/}}
\fi

\section{Discussion and Limitations}

In this section I will conclude and discuss limitations

\subsection{Discussion}

Talk about meshes, radiance fields etc.

\subsection{Applications in Animal Tracking}

Discussion as to what GSK have been doing.

\subsection{Future Work}

What needs to happen etc.

% \section{Broader impact}

% Our method improves the ability of machines to understand human body poses in images and videos.
% Understanding people automatically may arguably be misused by bad actors.
% However, importantly, our method is \emph{not} a form of biometric as it does \emph{not} allow the identification of people.
% Rather, only their overall body shape and pose is reconstructed, but these details are insufficient for unique identification.
% In particular, individual facial features are not reconstructed at all.

% Furthermore, our method is an improvement of existing capabilities, but does not introduce a radical new capability in machine learning.
% Thus our contribution is unlikely to facilitate misuse of technology which is already available to anyone.

% Finally, any potential negative use of a technology should be balanced against positive uses.
% Understanding body poses has many legitimate applications in VR and AR, medical, assistance to the elderly, assistance to the visual impaired, autonomous driving, human-machine interactions, image and video categorization, platform integrity, etc.



