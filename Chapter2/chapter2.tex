%!TEX root = ../thesis.tex
%*******************************************************************************
%****************************** Second Chapter *********************************
%*******************************************************************************

\chapter{Related Work}

\ifpdf
    \graphicspath{{Chapter2/Figs/Raster/}{Chapter2/Figs/PDF/}{Chapter2/Figs/}}
\else
    \graphicspath{{Chapter2/Figs/Vector/}{Chapter2/Figs/}}
\fi

% Awf: Do you really think it's a good idea to go into e.g. normalizing flows which are only relevant in the final chapter? It will make those final chapters quite short otherwise. Look at Kendall's thesis: file:///C:/Users/bjb10042/Downloads/thesis.pdf#page=27&zoom=100,116,581

% Awf: I wanted to do some toy experiments to help explain the effect of e.g. silhouette loss. Does that belong in this related work section?

\section{Introduction}

In this section, I will introduce the components required for the rest of this thesis.

\section{Representing 3D Objects}

Talk about meshes, radiance fields etc.

\section{Reconstructing 3D Objects from Images}

\subsection{Classical theory}

Give an overview towards classical NRSFM methods. Introduce camera geometry, rendering, etc. include lots of maths.

\subsection{Articulated Subjects}

Articulated subjects pose more challenges etc. Some methods maintain a model-free approach as before, talk about various works for humans (e.g. PiFu).

However, model-based approaches offer certain advantages (semantically meaningful parts, etc.). Explain SMPL and SMAL models.

\section[Template-Based]{3D Reconstruction with Deformable Template Meshes}

Introduce energy based methods and include loss functions etc., incl. SMAL optimizer, SMALR and finally SMAL-ST (deep learning techniques). All the SMPL approaches: HMR, GraphCMR, SPIN etc. 

Explain the various types of loss that are used. Good place for a Andrew's table.

\section{Data-Limited Training}

What do people do when there is no data available?

\section{Dealing with Ambiguous Input}

How to people cope with ambiguities?








