

%!TEX root = ../thesis.tex
%*******************************************************************************
%****************************** Third Chapter **********************************
%*******************************************************************************
\chapter{Learning from Synthetic Data; Bridging the Domain Gap}\label{chap:cgas}

\def\figref#1{Fig.~\ref{fig:#1}}

% **************************** Define Graphics Path **************************
\ifpdf
    \graphicspath{{Chapter4/Figs/Raster/}{Chapter4/Figs/PDF/}{Chapter4/Figs/}}
\else
    \graphicspath{{Chapter4/Figs/Vector/}{Chapter4/Figs/}}
\fi

% Plan:
% - Introduction - complain about the non-automatic equivalent systems & lack of training data for animals. Impractical to obtain. 
% - Begin with a discussion on systems that use synthetic data for training. Related work should focus on these
% - Discuss methods for generating accurate synthetic data, particularly focus on SURREAL. 
% - Create experiments in which different methods are trained and tested on varying levels of fake data.
% - In-depth explanation of synthetic data creation
% - In-depth explanation of training the joint predictor. Possibly ablate the sampling strategy / architecture. Diagrams etc.
% - Explain why we need to fix with a QP/GA.
% - In depth formulation of the QP and show it works, then explain the GA.
% - Experimental evaluation -- in depth explanation about the data collection exercise etc.
% - Experimental comparison between the two methods
% - In-depth explantion of the SMAL optimizer.
% - Results and comparison to related work
% - Summary 

% Extra experiments: 
% (1) Show that training on synthetic real data doesn't work (e.g. fake textures).
% (2) Use a GAN to generate synthetic data? Why not? Articulated sections didn't work.

\section{Introduction}\label{s:intro}

We are interested in reconstructing 3D human pose from the observation of single 2D images.
As humans, we have no problem in predicting, at least approximately, the 3D structure of most scenes, including the pose and shape of other people, even from a single view.
However, 2D images notoriously~\citep{Faugeras01geometry} do not contain sufficient geometric information to allow recovery of the third dimension.
Hence, single-view reconstruction is only possible in a probabilistic sense and the goal is to make the posterior distribution as sharp as possible, by learning a strong prior on the space of possible solutions.

Recent progress in single-view 3D pose reconstruction has been impressive.
Methods such as HMR~\citep{kanazawa18end-to-end}, GraphCMR~\citep{kolotouros19convolutional} and SPIN~\citep{kolotouros19learning} formulate this task as learning a deep neural network that maps 2D images to the parameters of a 3D model of the human body, usually SMPL~\cite{loper15smpl}.
These methods work well in general, but not always~(\cref{fig:issues}).
Their main weakness is processing \emph{heavily occluded images} of the object.
When a large part of the object is missing, say the lower body of a sitting human, they output reconstructions that are often implausible.
Since they can produce only one hypothesis as output, they very likely learn to approximate the mean of the posterior distribution, which may not correspond to any plausible pose.
Unfortunately, this failure modality is rather common in applications due to scene clutter and crowds.

In this paper, we propose a solution to this issue.
Specifically, we consider the challenge of recovering 3D mesh reconstructions of complex articulated objects such as humans from highly ambiguous image data, often containing significant occlusions of the object.
Clearly, it is generally impossible to reconstruct the object uniquely if too much evidence is missing; however, we can still predict a \emph{set} containing all possible reconstructions (see \cref{fig:splash}), making this set as small as possible.
While ambiguous pose reconstruction has been previously investigated, as far as we know, this is the first paper that looks specifically at a deep learning approach for ambiguous reconstructions of the \emph{full human mesh}.

% splash
\begin{figure}
\setlength{\fboxsep}{0pt}%
\setlength{\fboxrule}{0pt}%
\centering{\begin{tabular}{@{}c@{}}
    \includegraphics[width=0.49\linewidth,trim=4 8 8 10,clip]{splash/sample_2.pdf} \includegraphics[width=0.49\linewidth,trim=4 8 8 10,clip]{splash/sample_7.pdf}\\
    \includegraphics[width=0.49\linewidth,trim=8 10 10 12,clip]{splash/sample_8.pdf} \includegraphics[width=0.49\linewidth,trim=8 10 10  10,clip]{splash/sample_13.pdf}
\end{tabular}}
\vspace{-0.2cm}
\captionof{figure}{
\textbf{Human mesh recovery in an ambiguous setting.}
We propose a novel method that, given an occluded input image of a person, outputs the set of meshes which constitute plausible human bodies that are consistent with the partial view.
The ambiguous poses are predicted using a novel $n$-quantized-best-of-$M$ method.\label{fig:splash}}
% \vspace{-2cm}
\end{figure}

Our primary contribution is to introduce a principled multi-hypothesis framework to model the ambiguities in monocular pose recovery.
In the literature, such multiple-hypotheses networks are often trained with a so-called \emph{best-of-$M$} loss --- namely, during training, the loss is incurred only by the best of the $M$ hypothesis, back-propagating gradients from that alone~\cite{guzman2012multiple}.
In this work we opt for the \emph{best-of-$M$} approach since it has been show to outperform  alternatives (such as variational auto-encoders or mixture density networks) in tasks that are similar to our 3D human pose recovery, and which have constrained output spaces \cite{rupprecht17learning}.

\begin{wrapfigure}{r}{0.4\textwidth}
  \vspace{-0.3cm}
  \begin{center}
    \includegraphics[width=\linewidth]{failures/failure_summary_v2} %
  \end{center}
    \vspace{-0.3cm}
    \caption{\textbf{Top}: Pretrained SPIN model tested on an ambiguous example, \textbf{Bottom}: SPIN model after fine-tuning to ambiguous examples. Note the network tends to regress to the mean over plausible poses, shown by predicting the missing legs vertically downward --- arguably the average position over the training dataset.}\label{fig:issues}
    % \vspace{-0.3cm}
\end{wrapfigure}

% We also make an important contribution to better optimize our multi-hypothesis predictions.

A major drawback of the \emph{best-of-$M$} approach is that it only guarantees that \emph{one} of the hypotheses lies close to the correct solution; however, it says nothing about the plausibility, or lack thereof, of the \emph{other} $M-1$ hypotheses, which can be arbitrarily `bad'.%
%
\footnote{
Theoretically, best-of-$M$ can minimize its loss by quantizing optimally (in the sense of minimum expected distortion) the posterior distribution, which would be desirable for coverage.
However, this is \emph{not} the only solution that optimizes the best-of-$M$ training loss, as in the end it is sufficient that \emph{one} hypothesis per training sample is close to the ground truth.
In fact, this is exactly what happens; for instance, during training hypotheses in best-of-$M$ are known to easily become degenerate and `die off', a clear symptom of this problem.
}
%
Not only does this mean that most of the hypotheses may be uninformative, but in an application we are also unable to tell \emph{which} hypothesis should be used, and we might very well pick a `bad'
one.
% as there is no way of selecting the best one.
This has also a detrimental effect during learning because it  makes gradients sparse as prediction errors are back-propagated only through one of the $M$ hypotheses for each training image.

In order to address these issues, our first contribution is a \emph{hypothesis reprojection loss} that forces each member of the multi-hypothesis set to correctly reproject to 2D image keypoint annotations.
The main benefit is to constrain the \emph{whole} predicted set of meshes to be consistent with the observed image, not just the best hypothesis, also addressing gradient sparsity.

Next, we observe that another drawback of the best-of-{$M$} pipelines is to be tied to a particular value of $M$, whereas in applications we are often interested in tuning the number of hypothesis considered.
Furthermore, minimizing the reprojection loss makes hypotheses geometrically consistent with the observation, but not necessarily likely.
Our second contribution is thus to improve the flexibility of best-of-$M$ models by allowing them to output any smaller number $n<M$ of hypotheses while at the same time making these hypotheses \emph{more representative of likely} poses.
The new method, which we call $n$-quantized-best-of-$M$, does so by quantizing the best-of-$M$ model to output weighed by a \emph{explicit pose prior}, learned by means of normalizing flows.

% In order to do so, since best-of-$M$ lacks an explicit n\emph{prior} over plausible human body poses, thus resulting in an optimal coverage of the \emph{likely} solutions.

%an arbitrary number of $n<M$ hypotheses that optimally cover the set of plausible poses.

% Compared to prior work, we also make a simple but important contribution of modelling ambiguity in the space of 3D model parameters.
% Existing approaches, such as Mixture Density Networks~\cite{bishop94mixture,li19generating}, output instead a distribution on the reconstructed 3D location of a finite set of human body joints.
% This is straightforward, but difficult to extend to full 3D meshes.
% Instead, we argue that the parameters of the 3D model offer a better space for coding not just 3D shapes and poses, but also their ambiguities.
% As far as we could determine, we are the first to model ambiguous reconstructions directly in the space of human body model parameters.

To summarise, our key contributions are as follows.
First, we deal with the challenge of 3D mesh reconstruction for articulated objects such as humans in \emph{ambiguous} scenarios.
Second, we introduce a \emph{$n$-quantized-best-of-$M$} mechanism to allow best-of-$M$ models to generate an arbitrary number of $n<M$ predictions.
Third, we introduce a mode-wise re-projection loss for multi-hypothesis prediction, to ensure that predicted hypotheses are \emph{all} consistent with the input.

Empirically, we achieve state-of-the-art monocular mesh recovery accuracy on Human36M, its more challenging version augmented with heavy occlusions, and the 3DPW datasets.
Our ablation study validates each of our modelling choices, demonstrating their positive effect.

\section{Preliminaries}

\subsection{Deformable 3D quadruped model}

This section provides a formal definition for the deformable 3D model that is used to generate synthetic training data and in the model fitting stage to obtain the final mesh. Our system assumes a deformable 3D model such as SMAL~\cite{zuffi2017menagerie} which parametrizes a 3D mesh as a function of {\em pose} parameters~$\pose \in \R\npose$ (e.g.\ joint angles) and {\em shape} parameters~$\shape \in \R\nshape$. 
As discussed in \Cref{chap:relwork}, a 3D mesh is an array of vertices $\verts \in \RR 3\nverts$ (the vertices are columns of a $3 \times \nverts$ matrix) and a set of triangles represented as integer triples $(i,j,k)$, which are indices into the vertex array.
A deformable model such as SMAL may be viewed as supplying a set of triangles, and a function
\begin{equation}
\verts(\pose, \shape) : \R \npose \times \R \nshape \mapsto \RR 3 \nverts
\end{equation}
which generates the 3D model for a given pose and shape.
The mesh topology (i.e.~the triangle vertex indices) is provided by the deformable model, and is the same for all shapes and poses we consider, so in the sequel a mesh will be defined only by the 3D positions of its vertices.

In any given image, the model's 3D {\em position} (i.e.\ translation and orientation) is also unknown, and will be represented by a parametrization $\posn$ which may be for example translation as a 3-vector and rotation in axis angle form. Application of such a transformation to a $3\times\nverts$ matrix will be denoted by $*$, so that 
\begin{equation}
\posn * \verts(\pose, \shape)
\end{equation}
represents a 3D model of given pose and shape transformed to its 3D position.

It is also necessary to define a model's {\em joints}.  These appear naturally in models with an explicit skeleton, but more generally they can be defined as some function mapping from the model parameters to an array of 3D points analogous to the vertex transformation above. Note that even in the case of rigged models, this provides a mechanism to add additional joints beyond the ones required to drive model deformation. In any case, joints are defined by post-multiplying by a $\nverts \times \njoints$ matrix $\jointselect$.  The $j^{\text{th}}$ column of~$\jointselect$ defines the 3D position of joint~$j$ as a linear combination of the vertices (this is quite general, as $\verts$ may include vertices not mentioned in the triangulation).  

\subsection{Camera model, joint reprojection and silhouette rendering}
For both synthetic image generation and the later model fitting stage, it is necessary to be able to \emph{render} the 3D model. A general camera model is described by a function $\proj: \R{3} \mapsto \R{2}$.  This function incorporates details of the camera intrinsics such as focal length, which are assumed known.  
Thus 
\begin{equation} \label{eq:project_joints}
\kappa(\posn, \pose, \shape) := \proj(\posn * \verts(\pose, \shape) \jointselect)
\end{equation}
is the $2\times \njoints$ matrix whose columns are 2D joint locations corresponding to a 3D model specified by (position, pose, shape) parameters $(\posn, \pose, \shape)$.

The model is also assumed to be supplied with a rendering function $R$ which takes a vertex array in camera coordinates, and generates a 2D binary image of the model silhouette.  That is,
\begin{equation} \label{eq:render_sil}
R\bigl(\posn * \verts(\pose, \shape)\bigr) \in \mathbb{B}^{W\times H}
\end{equation}
for an image resolution of $W \times H$.  In order to allow derivatives to be propagated through $R$ (essential for the silhouette term in the model fitting stage), the \emph{differentiable renderer} of Loper et al.~\cite{loper2014opendr} is used. Please see the relevant section in \Cref{chap:relwork} for further details.

\subsection{System overview}

The test-time problem to be solved is to take a sequence of input images
\[
\mathcal{I} = \seq{I}{t}{1}{T}
\]
which are segmented to the silhouette of a single animal (i.e.~a video with multiple animals is segmented multiple times), producing a sequence of binary silhouette images 
\[
\mathcal{S} = \seq{S}{t}{1}{T}.
\]

The computational task is to output for each input image the shape, pose, and position parameters describing the animal's motion. Inspired by recent work in human 3D reconstruction, this objective can be broken down into a multiple stage pipeline. 

The core components are as follows:

\begin{enumerate}
    \item The discriminative front-end extracts silhouettes from video, and then uses the silhouettes to predict multi-modal heatmaps, from which 2D joint positions are obtained with multiple candidates per joint. 
    \item Optimal joint assignment (OJA) corrects confused or missing skeletal predictions by finding an optimal assignment of joints from a set of network-predicted proposals. 
    \item Generative deformable 3D model is fitted to the silhouettes and joint candidates as an energy minimization process.
\end{enumerate}

These stages are described in detail over the following three sections.


\section{Predicting 2D joint candidate locations}

% Explain hourglass network, it's structure and why it's good.
The goal of the first stage is to take, for each video frame, an image reperesenting the animal and to output a $W \times H \times \njoints$ tensor of heatmaps. To achieve this, we train the stacked hourlgass network~\cite{newell2016stacked} of Newell et al. to a large dataset of synthetically generated quadruped images. 

\subsection{Generating synthetic quadruped images}

In order to render a synthetic quadruped image, a set of pose $\pose$, shape $\shape$ and position $\posn$ parameters are required. With these parameters, a (textureless) training image can be generated by applying \Cref{eq:render_sil} and corresponding ground truth 2D joints locations can be generated via \Cref{eq:project_joints}. What remains is to ensure a model trained on the synthetic images generalizes to the real-world test images. To achieve this, it is important that the training dataset captures the modes of variation by appropriately sampling the model parameters. 

\subsubsection{Sampling shape and pose parameters}

Given that the real-world test images exhibit multiple quadruped species, a primary mode of variation is the animal's \emph{shape} characterisitics. Naively, generating a dataset large enough to capture possible test animal shapes would be a challenging task, since it would require access to multiple real or artist-generated 3D scans. Fortunately, the SMAL model defines a linear shape space which allows repeated sampling of different (and realistic) quadruped shapes. Of course, having been built from only toy figurines, the variation is still somewhat limited, but as the later experimental section will show, the model is fit for purpose. Synthetic shape parameters $\shape$ are obtained by sampling from a Gaussian prior:

\begin{equation}\label{eq:sampling_shape}
X = Y
\end{equation}

Another mode of variation in test images is the various animal limb positions. These can again be synthetised by sampling from a gaussian prior constructed from a dataset of animal poses. Since no such dataset exists for real animals, a prior is instead built from a small set of artist-generated poses, originally provided by the SMAL model authors. The sampling strategy is therefore given by:


\begin{equation}\label{eq:sampling_pose}
    X = y
\end{equation}


\subsubsection{Sampling position (camera) parameters}

The random camera positions are generated as follows: the orientation of the camera relative to the animal is uniform in the range $[0, 2\pi]$, the distance from the animal is uniform in the range 1 to 20 meters and the camera height is in the range $[0,\frac{\pi}{2}]$. This smaller range is chosen to restrict unusual camera elevation. Finally, the camera ``look" vector is towards a point uniformly in a 1m cube around the animal's center, and the ``up" vector is Gaussian around the model Y axis.

\subsubsection{What about textures and backgrounds?} 

At this point, the synthetic test images have plausible shape, pose and positional parameters but are lacking in realistic texture. Furthermore, it is not clear how to render plausible background scenes, nor how to convincingly place the 3D animal in such a scene. This challenge allows for a primary contribution of this work: using the silhouette domain as a representation much easier to synthetise (since surface texture maps and background textures are lost) and can be obtained from real-world test images. Another important characteristic of the mapping between real-world images and silhouette counterparts is that the shape and pose information is mostly preserved, apart from a few ambiguities (e.g. limb ordering due to lacking interior contours). 

To summarise, training data comprises $(S, \kappa)$ pairs, that is pairs of binary silhouette images, and the corresponding 2D joint locations as a $2\times J$ matrix. See an example in   To generate each image, a random shape vector $\shape$, pose parameters $\pose$ and camera position $\posn$ are drawn, and used to render a silhouette $R\bigl(\posn * \verts(\pose, \shape)\bigr)$ and 2D joint locations $\kappa(\posn,\pose,\shape)$.


% The corresponding 2D joint positions, represented as a $2\times \njoints$ matrix are given, as above, by
% \[
%     \kappa(\posn, \pose, \shape) := \proj(\posn * \verts(\pose, \shape) \jointselect)
% \]

\subsection{Prediction of 2D joint locations using multimodal heatmaps}

% explain that the quality of the joint prediction, while good still leaves unsatisfactory predictions which can cause problems to the later energy minimization steps. To overcome this, 


A pose estimation network can now be trained on a dataset of binary silhouette images $S$ with corresponding 2D joint locations $\kappa(\posn,\pose,\shape)$. The core network architecture used for this task is the stacked hourglass network~\cite{newell2016stacked} of Newell et al., with an adaptation to produce multi-modal outputs. The stacked hourglass network is now briefly described:

\subsubsection{Stacked hourglass network}

A stacked hourglass network is a convolutional neural network specifically designed for the task of pose estimation. Hourglass allows inference to take place across multiple scales; an important advantage allowing the network to reason about global information (such as the full body) and local information (such as face features). This is achieved using a repeated bottom-up, top-down modules which each produce a $W \times H \times \njoints$ heatmap tensor. Supervision is applied to each of these heatmaps, which are then passed to ths subsequent block, allowing high level features to be reevaluated for higher order spatial relationships (generally only present at low resolutions). The network's architecture can be seen in detail in Figure XXX. Hourglass uses a mean squared error loss against a ground truth heatmap tensor, which is applied to the network's final output and all intermediatary hetamaps. Therefore, ground truth joint coordinates $\kappa(\posn,\pose,\shape)$ must first be encoded into a $W \times H \times \njoints$ tensor of heatmaps. The original authors find that the network has better convergence properties if the ground truth heatmaps are blurred slightly, since a gradient is then provided to predicted ``near misses''. This is achieved by blurring ground truth heatmaps with a Gaussian kernel of radius $\sigma$. The final 2D joint positions are then obtained using non-maximum suppression on the output heatmaps

\begin{equation}\label{eq:non-max-suppression}
    A = B    
\end{equation}

\subsubsection{Adaptations for training on synthetic data}

This setup is mostly suitable for training on synthetic quadruped data, subject to a couple of adaptations. Firstly, the silhouette training images differ considerably in appearance to the RGB images used by the original Hourglass authors. A common difficulty for training pose estimation networks on full RGB images is in trying to distinguish between objects and the background class. With binary silhouette images, this distinction is made trivial as background and foreground are assigned values $0$ or $1$ respectively. Unfortunately, this causes a problem when naively training HourglassNet on the generated synthetic images as the network can greatly minimize the loss by simply predicting `background' for every image pixel (since background pixels tend to outnumber foreground pixels). The fact that `background' is a much simpler class to predict than the other joint classes causes a training instability which is challenging to overcome by adjusting the learning rate. Instead, the loss function is replaced with a \emph{weighted} version of the mean squared error loss. This modification encourages the network to devote it's attention to all classes equally. The precise formulation for this is given as follows:

\begin{equation}\label{eq:weighted-mse}
    A = B
\end{equation}


\subsubsection{Adapting stacked hourglass network for multi-output learning}

As detailed in the later experimental section, the network trained using this process generalizes well from synthetic to real images due to the use of the silhouette, and produces accurate predictions for most joints. However, the predictor performs poorly on some joints due to ambiguities which result from the lack of interior contours in the silhouette input data. These missing details cause often result in confusion between joint ``aliases'': left and right or front and back legs.  When these predictions are wrong and are represented by high confidence heatmap regions, little probability mass is assigned to the area around the correct leg, meaning no available proposal is present after non-maximal suppression.

This is handled with a further technical contribution; handling the prediction uncertainty by adapting the stacked hourglass network to produce multiple outputs. This is achieved by explicitly training the network to assign some probability mass to the ``aliased'' joints. For each joint, a list of potential aliases are defined as weights $\lambda_{j,j'}$ and linearly blend the unimodal heatmaps $G$ to give the final training heatmap $H$:

\begin{equation}
    H_{j}(p) = \sum_{j'} \lambda_{j,j'} G(p; \kappa_{j'}, \sigma)
\end{equation}

For non-aliased joints $j$ (all but the legs), $\lambda_{j,j} = 1$ and $\lambda_{j,j'} = 0$, yielding the unimodal maps. For aliased joints, the joint is assigned a weight $\lambda_{j,j} = 0.75$ and aliases are assigned a weight $\lambda_{j,j'} = 0.25$. This ratio is shown to ensure opposite legs have sufficient probability mass to pass through a modest non-maximal suppression threshold without overly biasing the skeleton with maximal predicted confidence. An example of a heatmap predicted by a network trained on multimodal training samples is illustrated in \figref{single_multi}. Note that the construction of at most bi-modal ground truth heatmaps sets a practical constraint on the number of output modes. In other words, the loss is minimized if the network produces one output mode for non-aliased joints and two output modes for aliased joints. 

% % Please add the following required packages to your document preamble:
% \usepackage{booktabs}
\begin{table}[]
    \begin{tabular}{@{}lllll@{}}
    \toprule
    Joint ID & Joint Name                    & Weight & Alias                         & Alias Weight \\ \midrule
    0        & Left front leg: paw           & 0.75   & Right front leg: paw          & 0.25         \\
    1        & Left front leg: middle joint  & 0.75   & Right front leg: middle joint & 0.25         \\
    2        & Left front leg: top           & 0.75   & Right front leg: top          & 0.25         \\
    3        & Left rear leg: paw            & 0.75   & Right rear leg: paw           & 0.25         \\
    4        & Left rear leg: middle joint   & 0.75   & Right rear leg: middle joint  & 0.25         \\
    5        & Left rear leg: top            & 0.75   & Right rear leg: top           & 0.25         \\
    6        & Right front leg: paw          & 0.75   & Left front leg: paw           & 0.25         \\
    7        & Right front leg: middle joint & 0.75   & Left front leg: middle joint  & 0.25         \\
    8        & Right front leg: top          & 0.75   & Left front leg: top           & 0.25         \\
    9        & Right rear leg: paw           & 0.75   & Left rear leg: paw            & 0.25         \\
    10       & Right rear leg: middle joint  & 0.75   & Left rear leg: middle joint   & 0.25         \\
    11       & Right rear leg: top           & 0.75   & Left rear leg: top            & 0.25         \\
    12       & Tail start                    & 1.0    &                               &              \\
    13       & Tail end                      & 1.0    &                               &              \\
    14       & Base of left ear              & 1.0    &                               &              \\
    15       & Base of right ear             & 1.0    &                               &              \\
    16       & Nose                          & 1.0    &                               &              \\
    17       & Chin                          & 1.0    &                               &             
    \end{tabular}
\end{table}\label{tab:joint_weights}

\begin{figure}[t]
% \begin{floatrow}
% \ffigbox{%%%%%%%%%%%%
\centering
\begin{tabular}{cc}
\includegraphics[trim={4cm 10cm 4cm 10cm},clip,width=0.5\linewidth]{single_vs_multi_new/left_heatmap_single.png} &
\includegraphics[trim={4cm 10cm 4cm 10cm},clip,width=0.5\linewidth]{single_vs_multi_new/left_heatmap_multi.png} \\
\end{tabular}
% }
{
\caption{Example predictions from a network trained on unimodal (top) and multi-modal (bottom) ground-truth  for front-left leg joints.}
\label{fig:single_multi}
}
% \end{floatrow}
\end{figure}

\begin{figure}[t]
% \begin{floatrow}
% \ffigbox{
\centering
\def\lp#1#2{\labelledpic{\includegraphics[width=0.20\linewidth]{#2}}{#1}}
\begin{tabular}{cccc}
\lp a{skeletons_new/skeleton_rgb_dog_cropped.jpg}&
\lp b{skeletons_new/skeleton_rgb_impala_cropped.jpg}&
\lp c{skeletons_new/skeleton_rgb_rhino_cropped.jpg}&
\lp d{skeletons_new/skeleton_rgb_horsejump-high_cropped.jpg}\\
\end{tabular}
% }
{
\caption{Example outputs from the joint prediction network, with maximum likelihood predictions linked into skeleton.}
\label{fig:exp-network}
}
% \end{floatrow}
\end{figure}
\input{Chapter4/oja.tex}
\input{Chapter4/model-fitting.tex}
\input{Chapter4/experiments.tex}
\section{Conclusions}
The chapter introduces a technique for reconstructing 3D quadrupeds from video by using a quadruped model parameterized in shape and pose. By incorporating automatic segmentation tools, the pipeline can be deployed without requiring human intervention or even precise knowledge of the species of animal being considered. The method performs well on examples encountered in the real world, generalizes to unseen animal species and is robust to challenging shapes and poses. 

As a direction for future work, it would be worthwhile to look at methods for synthetic image generation that preserve important edge information lost during silhouette extraction. In particular, the lost interior contours cause ambiguities which necessitate the OJA method described here. An alternative is to extend the recent work of SMALST~\lazycite{SMALST}{SMALST} (which requires hand-clicked training images) by instead synthetizing realistic textures using generative adverserial networks. Of course, it is important to ensure the texture generation process preserves the sampled pose parameters. A naive method is to rendering texture as a UV map on top of the SMAL model, but this could also be framed as an image-to-image translation problem, beginning with a low detail SMAL render and mapping to a photorealistic image with a carefully designed pose-preserving generator. Other components of the system which could be improved would be building a more sophisticated motion prior, able to represent likely animal trajectories. In addition, robustness to environmental factors as shown in \Cref{fig:blooper} could be improved by rendering synthetic causes of occlusion (perhaps even as rectangles). 
    